\subsection{噪声层 Noise}
    
\subsubsection{GaussianNoise {\href{https://github.com/keras-team/keras/blob/master/keras/layers/noise.py\#L14}{{[}source{]}}}}

\begin{Shaded}
\begin{Highlighting}[]
\NormalTok{keras.layers.GaussianNoise(stddev)}
\end{Highlighting}
\end{Shaded}

应用以 0 为中心的加性高斯噪声。

这对缓解过拟合很有用 (你可以将其视为随机数据增强的一种形式)。
高斯噪声(GS)是对真实输入的腐蚀过程的自然选择。

由于它是一个正则化层,因此它只在训练时才被激活。

\textbf{参数}

\begin{itemize}
\tightlist
\item
  \textbf{stddev}: float,噪声分布的标准差。
\end{itemize}

\textbf{输入尺寸}

可以是任意的。 如果将该层作为模型的第一层,则需要指定
\texttt{input\_shape} 参数 (整数元组,不包含样本数量的维度)。

\textbf{输出尺寸}

与输入相同。




\subsubsection{GaussianDropout {\href{https://github.com/keras-team/keras/blob/master/keras/layers/noise.py\#L58}{{[}source{]}}}}

\begin{Shaded}
\begin{Highlighting}[]
\NormalTok{keras.layers.GaussianDropout(rate)}
\end{Highlighting}
\end{Shaded}

应用以 1 为中心的 乘性高斯噪声。

由于它是一个正则化层,因此它只在训练时才被激活。

\textbf{参数}

\begin{itemize}
\tightlist
\item
  \textbf{rate}: float,丢弃概率(与 \texttt{Dropout} 相同)。
  这个乘性噪声的标准差为 \texttt{sqrt(rate\ /\ (1\ -\ rate))}。
\end{itemize}

\textbf{输入尺寸}

可以是任意的。 如果将该层作为模型的第一层,则需要指定
\texttt{input\_shape} 参数 (整数元组,不包含样本数量的维度)。

\textbf{输出尺寸}

与输入相同。

\textbf{参考文献}

\begin{itemize}
\tightlist
\item
  \href{http://www.cs.toronto.edu/~rsalakhu/papers/srivastava14a.pdf}{Dropout:
  A Simple Way to Prevent Neural Networks from Overfitting Srivastava,
  Hinton, et al. 2014}
\end{itemize}




\subsubsection{AlphaDropout {\href{https://github.com/keras-team/keras/blob/master/keras/layers/noise.py\#L105}{{[}source{]}}}}

\begin{Shaded}
\begin{Highlighting}[]
\NormalTok{keras.layers.AlphaDropout(rate, noise_shape}\OperatorTok{=}\VariableTok{None}\NormalTok{, seed}\OperatorTok{=}\VariableTok{None}\NormalTok{)}
\end{Highlighting}
\end{Shaded}

将 Alpha Dropout 应用到输入。

Alpha Dropout是一种
\texttt{Dropout},它保持输入的平均值和方差与原来的值不变, 已在 dropout
之后仍然保证数据的自规范性。 通过随机将激活设置为负饱和值,Alpha Dropout
非常适合按比例缩放的指数线性单元(SELU)。

\textbf{参数}

\begin{itemize}
\tightlist
\item
  \textbf{rate}: float,丢弃概率(与 \texttt{Dropout} 相同)。
  这个乘性噪声的标准差为 \texttt{sqrt(rate\ /\ (1\ -\ rate))}。
\item
  \textbf{seed}: 用作随机种子的 Python 整数。
\end{itemize}

\textbf{输入尺寸}

可以是任意的。 如果将该层作为模型的第一层,则需要指定
\texttt{input\_shape} 参数 (整数元组,不包含样本数量的维度)。

\textbf{输出尺寸}

与输入相同。

\textbf{参考文献}

\begin{itemize}
\tightlist
\item
  \href{https://arxiv.org/abs/1706.02515}{Self-Normalizing Neural
  Networks}
\end{itemize}
\newpage