\subsection{核心网络层}
\subsubsection{Dense {\href{https://github.com/keras-team/keras/blob/master/keras/layers/core.py\#L743}{{[}source{]}}}}

\begin{Shaded}
\begin{Highlighting}[]
\NormalTok{keras.layers.Dense(units, activation}\OperatorTok{=}\VariableTok{None}\NormalTok{, use_bias}\OperatorTok{=}\VariableTok{True}, \\
\hspace{3.5cm}\NormalTok{kernel_initializer}\OperatorTok{=}\StringTok{'glorot_uniform'}\NormalTok{, bias_initializer}\OperatorTok{=}\StringTok{'zeros'}, \\
\hspace{3.5cm}\NormalTok{kernel_regularizer}\OperatorTok{=}\VariableTok{None}\NormalTok{, bias_regularizer}\OperatorTok{=}\VariableTok{None}, \\
\hspace{3.5cm}\NormalTok{activity_regularizer}\OperatorTok{=}\VariableTok{None}\NormalTok{, kernel_constraint}\OperatorTok{=}\VariableTok{None}\NormalTok{, bias_constraint}\OperatorTok{=}\VariableTok{None}\NormalTok{)}
\end{Highlighting}
\end{Shaded}

就是普通的全连接层。

\texttt{Dense} 实现以下操作:
\texttt{output\ =\ activation(dot(input,\ kernel)\ +\ bias)} 其中
\texttt{activation} 是按逐个元素计算的激活函数,\texttt{kernel}
是由网络层创建的权值矩阵,以及 \texttt{bias} 是其创建的偏置向量 (只在
\texttt{use\_bias} 为 \texttt{True} 时才有用)。

\begin{itemize}
\tightlist
\item
  \textbf{注意}: 如果该层的输入的秩大于2,那么它首先被展平然后 再计算与
  \texttt{kernel} 的点乘。
\end{itemize}

\textbf{例}

\begin{Shaded}
\begin{Highlighting}[]
\CommentTok{# 作为 Sequential 模型的第一层}
\NormalTok{model }\OperatorTok{=} \NormalTok{Sequential()}
\NormalTok{model.add(Dense(}\DecValTok{32}\NormalTok{, input_shape}\OperatorTok{=}\NormalTok{(}\DecValTok{16}\NormalTok{,)))}
\CommentTok{# 现在模型就会以尺寸为 (*, 16) 的数组作为输入,}
\CommentTok{# 其输出数组的尺寸为 (*, 32)}

\CommentTok{# 在第一层之后,你就不再需要指定输入的尺寸了:}
\NormalTok{model.add(Dense(}\DecValTok{32}\NormalTok{))}
\end{Highlighting}
\end{Shaded}

\textbf{参数}

\begin{itemize}
\tightlist
\item
  \textbf{units}: 正整数,输出空间维度。
\item
  \textbf{activation}: 激活函数 (详见
  \hyperref[activations]{activations})。 若不指定,则不使用激活函数
  (即, ``线性''激活: \texttt{a(x)\ =\ x})。
\item
  \textbf{use\_bias}: 布尔值,该层是否使用偏置向量。
\item
  \textbf{kernel\_initializer}: \texttt{kernel} 权值矩阵的初始化器 (详见
  \hyperref[initializers]{initializers})。
\item
  \textbf{bias\_initializer}: 偏置向量的初始化器 (see
   \hyperref[initializers]{initializers}).
\item
  \textbf{kernel\_regularizer}: 运用到 \texttt{kernel}
  权值矩阵的正则化函数 (详见 \hyperref[regularizers]{regularizer})。
\item
  \textbf{bias\_regularizer}: 运用到偏置向的的正则化函数 (详见
  \hyperref[regularizers]{regularizer})。
\item
  \textbf{activity\_regularizer}: 运用到层的输出的正则化函数 (它的
  ``activation")。 (详见 \hyperref[regularizers]{regularizer})。
\item
  \textbf{kernel\_constraint}: 运用到 \texttt{kernel} 权值矩阵的约束函数
  (详见 \hyperref[constraints]{constraints})。
\item
  \textbf{bias\_constraint}: 运用到偏置向量的约束函数 (详见
  \hyperref[constraints]{constraints})。
\end{itemize}

\textbf{输入尺寸}

nD 张量,尺寸: \texttt{(batch\_size,\ ...,\ input\_dim)}。
最常见的情况是一个尺寸为 \texttt{(batch\_size,\ input\_dim)} 的 2D
输入。

\textbf{输出尺寸}

nD 张量,尺寸: \texttt{(batch\_size,\ ...,\ units)}。 例如,对于尺寸为
\texttt{(batch\_size,\ input\_dim)} 的 2D 输入, 输出的尺寸为
\texttt{(batch\_size,\ units)}。




\subsubsection{Activation {\href{https://github.com/keras-team/keras/blob/master/keras/layers/core.py\#L280}{{[}source{]}}}}


\begin{Shaded}
\begin{Highlighting}[]
\NormalTok{keras.layers.Activation(activation)}
\end{Highlighting}
\end{Shaded}

将激活函数应用于输出。

\textbf{参数}

\begin{itemize}
\tightlist
\item
  \textbf{activation}: 要使用的激活函数的名称 (详见:
  hyperref[activations]{activations}), 或者选择一个 Theano 或
  TensorFlow 操作。
\end{itemize}

\textbf{输入尺寸}

任意尺寸。 当使用此层作为模型中的第一层时, 使用参数
\texttt{input\_shape} (整数元组,不包括样本数的轴)。

\textbf{输出尺寸}

与输入相同。



\subsubsection{Dropout {\href{https://github.com/keras-team/keras/blob/master/keras/layers/core.py\#L78}{{[}source{]}}}}

\begin{Shaded}
\begin{Highlighting}[]
\NormalTok{keras.layers.Dropout(rate, noise_shape}\OperatorTok{=}\VariableTok{None}\NormalTok{, seed}\OperatorTok{=}\VariableTok{None}\NormalTok{)}
\end{Highlighting}
\end{Shaded}

将 Dropout 应用于输入。

Dropout 包括在训练中每次更新时, 将输入单元的按比率随机设置为 0,
这有助于防止过拟合。

\textbf{参数}

\begin{itemize}
\tightlist
\item
  \textbf{rate}: 在 0 和 1 之间浮动。需要丢弃的输入比例。
\item
  \textbf{noise\_shape}: 1D 整数张量, 表示将与输入相乘的二进制 dropout
  掩层的形状。 例如,如果你的输入尺寸为
  \texttt{(batch\_size,\ timesteps,\ features)},然后 你希望 dropout
  掩层在所有时间步都是一样的, 你可以使用
  \texttt{noise\_shape=(batch\_size,\ 1,\ features)}。
\item
  \textbf{seed}: 一个作为随机种子的 Python 整数。
\end{itemize}

\textbf{参考文献}

\begin{itemize}
\tightlist
\item
  \href{http://www.cs.toronto.edu/~rsalakhu/papers/srivastava14a.pdf}{Dropout:
  A Simple Way to Prevent Neural Networks from Overfitting}
\end{itemize}




\subsubsection{Flatten {\href{https://github.com/keras-team/keras/blob/master/keras/layers/core.py\#L465}{{[}source{]}}}}

\begin{Shaded}
\begin{Highlighting}[]
\NormalTok{keras.layers.Flatten()}
\end{Highlighting}
\end{Shaded}

将输入展平。不影响批量大小。

\textbf{例}

\begin{Shaded}
\begin{Highlighting}[]
\NormalTok{model }\OperatorTok{=} \NormalTok{Sequential()}
\NormalTok{model.add(Conv2D(}\DecValTok{64}\NormalTok{, }\DecValTok{3}\NormalTok{, }\DecValTok{3}\NormalTok{,}
                 \NormalTok{border_mode}\OperatorTok{=}\StringTok{'same'}\NormalTok{,}
                 \NormalTok{input_shape}\OperatorTok{=}\NormalTok{(}\DecValTok{3}\NormalTok{, }\DecValTok{32}\NormalTok{, }\DecValTok{32}\NormalTok{)))}
\CommentTok{# 现在:model.output_shape == (None, 64, 32, 32)}

\NormalTok{model.add(Flatten())}
\CommentTok{# 现在:model.output_shape == (None, 65536)}
\end{Highlighting}
\end{Shaded}


\subsubsection{Input {\href{https://github.com/keras-team/keras/blob/master/keras/engine/topology.py\#L1391}{{[}source{]}}}}

\begin{Shaded}
\begin{Highlighting}[]
\NormalTok{keras.engine.topology.Input()}
\end{Highlighting}
\end{Shaded}

\texttt{Input()} 用于实例化 Keras 张量。

Keras 张量是底层后端(Theano, TensorFlow or CNTK)
的张量对象,我们增加了一些特性,使得能够通过了解模型的输入
和输出来构建Keras模型。

例如,如果 a, b 和 c 都是 Keras 张量, 那么以下操作是可行的:
\texttt{model\ =\ Model(input={[}a,\ b{]},\ output=c)}

添加的 Keras 属性是: - \textbf{\texttt{\_keras\_shape}}: 通过
Keras端的尺寸推理 进行传播的整数尺寸元组。 -
\textbf{\texttt{\_keras\_history}}: 应用于张量的最后一层。
整个网络层计算图可以递归地从该层中检索。

\textbf{参数}

\begin{itemize}
\tightlist
\item
  \textbf{shape}: 一个尺寸元组(整数),不包含批量大小。A shape tuple
  (integer), not including the batch size. 例如,\texttt{shape=(32,)}
  表明期望的输入是按批次的 32 维向量。
\item
  \textbf{batch\_shape}: 一个尺寸元组(整数),包含批量大小。
  例如,\texttt{batch\_shape=(10,\ 32)} 表明期望的输入是 10 个 32
  维向量。 \texttt{batch\_shape=(None,\ 32)} 表明任意批次大小的 32
  维向量。
\item
  \textbf{name}: 一个可选的层的名称的字符串。
  在一个模型中应该是唯一的(不可以重用一个名字两次)。
  如未提供,将自动生成。
\item
  \textbf{dtype}: 输入所期望的数据类型,字符串表示 (\texttt{float32},
  \texttt{float64}, \texttt{int32}...)
\item
  \textbf{sparse}: 一个布尔值,指明需要创建的占位符是否是稀疏的。
\item
  \textbf{tensor}: 可选的可封装到 \texttt{Input} 层的现有张量。
  如果设定了,那么这个层将不会创建占位符张量。
\end{itemize}

\textbf{返回}

一个张量。

\textbf{例}

\begin{Shaded}
\begin{Highlighting}[]
\CommentTok{# 这是 Keras 中的一个逻辑回归}
\NormalTok{x }\OperatorTok{=} \NormalTok{Input(shape}\OperatorTok{=}\NormalTok{(}\DecValTok{32}\NormalTok{,))}
\NormalTok{y }\OperatorTok{=} \NormalTok{Dense(}\DecValTok{16}\NormalTok{, activation}\OperatorTok{=}\StringTok{'softmax'}\NormalTok{)(x)}
\NormalTok{model }\OperatorTok{=} \NormalTok{Model(x, y)}
\end{Highlighting}
\end{Shaded}


\subsubsection{Reshape {\href{https://github.com/keras-team/keras/blob/master/keras/layers/core.py\#L314}{{[}source{]}}}}

\begin{Shaded}
\begin{Highlighting}[]
\NormalTok{keras.layers.Reshape(target_shape)}
\end{Highlighting}
\end{Shaded}

将输入重新调整为特定的尺寸。

\textbf{参数}

\begin{itemize}
\tightlist
\item
  \textbf{target\_shape}: 目标尺寸。整数元组。 不包含表示批量的轴。
\end{itemize}

\textbf{输入尺寸}

任意,尽管输入尺寸中的所有维度必须是固定的。
当使用此层作为模型中的第一层时, 使用参数 \texttt{input\_shape}
(整数元组,不包括样本数的轴)。

\textbf{输出尺寸}

\texttt{(batch\_size,)\ +\ target\_shape}

\textbf{例}

\begin{Shaded}
\begin{Highlighting}[]
\CommentTok{# 作为 Sequential 模型的第一层}
\NormalTok{model }\OperatorTok{=} \NormalTok{Sequential()}
\NormalTok{model.add(Reshape((}\DecValTok{3}\NormalTok{, }\DecValTok{4}\NormalTok{), input_shape}\OperatorTok{=}\NormalTok{(}\DecValTok{12}\NormalTok{,)))}
\CommentTok{# 现在:model.output_shape == (None, 3, 4)}
\CommentTok{# 注意: `None` 是批表示的维度}

\CommentTok{# 作为 Sequential 模型的中间层}
\NormalTok{model.add(Reshape((}\DecValTok{6}\NormalTok{, }\DecValTok{2}\NormalTok{)))}
\CommentTok{# 现在: model.output_shape == (None, 6, 2)}

\CommentTok{# 还支持使用 `-1` 表示维度的尺寸推断}
\NormalTok{model.add(Reshape((}\OperatorTok{-}\DecValTok{1}\NormalTok{, }\DecValTok{2}\NormalTok{, }\DecValTok{2}\NormalTok{)))}
\CommentTok{# 现在: model.output_shape == (None, 3, 2, 2)}
\end{Highlighting}
\end{Shaded}


\subsubsection{Permute {\href{https://github.com/keras-team/keras/blob/master/keras/layers/core.py\#L413}{{[}source{]}}}}

\begin{Shaded}
\begin{Highlighting}[]
\NormalTok{keras.layers.Permute(dims)}
\end{Highlighting}
\end{Shaded}

根据给定的模式置换输入的维度。

在某些场景下很有用,例如将 RNN 和 CNN 连接在一起。

\textbf{例}

\begin{Shaded}
\begin{Highlighting}[]
\NormalTok{model }\OperatorTok{=} \NormalTok{Sequential()}
\NormalTok{model.add(Permute((}\DecValTok{2}\NormalTok{, }\DecValTok{1}\NormalTok{), input_shape}\OperatorTok{=}\NormalTok{(}\DecValTok{10}\NormalTok{, }\DecValTok{64}\NormalTok{)))}
\CommentTok{# 现在: model.output_shape == (None, 64, 10)}
\CommentTok{# 注意: `None` 是批表示的维度}
\end{Highlighting}
\end{Shaded}

\textbf{参数}

\begin{itemize}
\tightlist
\item
  \textbf{dims}: 整数元组。置换模式,不包含样本维度。 索引从 1 开始。
  例如, \texttt{(2,\ 1)} 置换输入的第一和第二个维度。
\end{itemize}

\textbf{输入尺寸}

任意。当使用此层作为模型中的第一层时, 使用参数 \texttt{input\_shape}
(整数元组,不包括样本数的轴)。

\textbf{输出尺寸}

与输入尺寸相同,但是维度根据指定的模式重新排列。



\subsubsection{RepeatVector {\href{https://github.com/keras-team/keras/blob/master/keras/layers/core.py\#L500}{{[}source{]}}}}

\begin{Shaded}
\begin{Highlighting}[]
\NormalTok{keras.layers.RepeatVector(n)}
\end{Highlighting}
\end{Shaded}

将输入重复 n 次。

\textbf{例}

\begin{Shaded}
\begin{Highlighting}[]
\NormalTok{model }\OperatorTok{=} \NormalTok{Sequential()}
\NormalTok{model.add(Dense(}\DecValTok{32}\NormalTok{, input_dim}\OperatorTok{=}\DecValTok{32}\NormalTok{))}
\CommentTok{# 现在: model.output_shape == (None, 32)}
\CommentTok{# 注意: `None` 是批表示的维度}

\NormalTok{model.add(RepeatVector(}\DecValTok{3}\NormalTok{))}
\CommentTok{# 现在: model.output_shape == (None, 3, 32)}
\end{Highlighting}
\end{Shaded}

\textbf{参数}

\begin{itemize}
\tightlist
\item
  \textbf{n}: 整数,重复次数。
\end{itemize}

\textbf{输入尺寸}

2D 张量,尺寸为 \texttt{(num\_samples,\ features)}。

\textbf{输出尺寸}

3D 张量,尺寸为 \texttt{(num\_samples,\ n,\ features)}。

\subsubsection{Lambda {\href{https://github.com/keras-team/keras/blob/master/keras/layers/core.py\#L542}{{[}source{]}}}}

\begin{Shaded}
\begin{Highlighting}[]
\NormalTok{keras.layers.Lambda(function, output_shape}\OperatorTok{=}\VariableTok{None}\NormalTok{, mask}\OperatorTok{=}\VariableTok{None}\NormalTok{, arguments}\OperatorTok{=}\VariableTok{None}\NormalTok{)}
\end{Highlighting}
\end{Shaded}

将任意表达式封装为 \texttt{Layer} 对象。

\textbf{例}

\begin{Shaded}
\begin{Highlighting}[]
\CommentTok{# 添加一个 x -> x^2 层}
\NormalTok{model.add(Lambda(}\KeywordTok{lambda} \NormalTok{x: x }\OperatorTok{**} \DecValTok{2}\NormalTok{))}
\end{Highlighting}
\end{Shaded}

\begin{Shaded}
\begin{Highlighting}[]
\CommentTok{# 添加一个网络层,返回输入的正数部分}
\CommentTok{# 与负数部分的反面的连接}

\KeywordTok{def} \NormalTok{antirectifier(x):}
    \NormalTok{x }\OperatorTok{-=} \NormalTok{K.mean(x, axis}\OperatorTok{=}\DecValTok{1}\NormalTok{, keepdims}\OperatorTok{=}\VariableTok{True}\NormalTok{)}
    \NormalTok{x }\OperatorTok{=} \NormalTok{K.l2_normalize(x, axis}\OperatorTok{=}\DecValTok{1}\NormalTok{)}
    \NormalTok{pos }\OperatorTok{=} \NormalTok{K.relu(x)}
    \NormalTok{neg }\OperatorTok{=} \NormalTok{K.relu(}\OperatorTok{-}\NormalTok{x)}
    \ControlFlowTok{return} \NormalTok{K.concatenate([pos, neg], axis}\OperatorTok{=}\DecValTok{1}\NormalTok{)}

\KeywordTok{def} \NormalTok{antirectifier_output_shape(input_shape):}
    \NormalTok{shape }\OperatorTok{=} \BuiltInTok{list}\NormalTok{(input_shape)}
    \ControlFlowTok{assert} \BuiltInTok{len}\NormalTok{(shape) }\OperatorTok{==} \DecValTok{2}  \CommentTok{# only valid for 2D tensors}
    \NormalTok{shape[}\OperatorTok{-}\DecValTok{1}\NormalTok{] }\OperatorTok{*=} \DecValTok{2}
    \ControlFlowTok{return} \BuiltInTok{tuple}\NormalTok{(shape)}

\NormalTok{model.add(Lambda(antirectifier,}
                 \NormalTok{output_shape}\OperatorTok{=}\NormalTok{antirectifier_output_shape))}
\end{Highlighting}
\end{Shaded}

\textbf{参数}

\begin{itemize}
\tightlist
\item
  \textbf{function}: 需要封装的函数。 将输入张量作为第一个参数。
\item
  \textbf{output\_shape}: 预期的函数输出尺寸。 只在使用 Theano
  时有意义。 可以是元组或者函数。 如果是元组,它只指定第一个维度;
  样本维度假设与输入相同:
  \texttt{output\_shape\ =\ (input\_shape{[}0{]},\ )\ +\ output\_shape}
  或者,输入是 \texttt{None} 且样本维度也是 \texttt{None}:
  \texttt{output\_shape\ =\ (None,\ )\ +\ output\_shape}
  如果是函数,它指定整个尺寸为输入尺寸的一个函数:
  \texttt{output\_shape\ =\ f(input\_shape)}
\item
  \textbf{arguments}: 可选的需要传递给函数的关键字参数。
\end{itemize}

\textbf{输入尺寸}

任意。当使用此层作为模型中的第一层时, 使用参数 \texttt{input\_shape}
(整数元组,不包括样本数的轴)。

\textbf{输出尺寸}

由 \texttt{output\_shape} 参数指定 (或者在使用 TensorFlow
时,自动推理得到)。


\subsubsection{ActivityRegularization {\href{https://github.com/keras-team/keras/blob/master/keras/layers/core.py\#L886}{{[}source{]}}}}

\begin{Shaded}
\begin{Highlighting}[]
\NormalTok{keras.layers.ActivityRegularization(l1}\OperatorTok{=}\FloatTok{0.0}\NormalTok{, l2}\OperatorTok{=}\FloatTok{0.0}\NormalTok{)}
\end{Highlighting}
\end{Shaded}

网络层,对基于代价函数的输入活动应用一个更新。

\textbf{参数}

\begin{itemize}
\tightlist
\item
  \textbf{l1}: L1 正则化因子 (正数浮点型)。
\item
  \textbf{l2}: L2 正则化因子 (正数浮点型)。
\end{itemize}

\textbf{输入尺寸}

任意。当使用此层作为模型中的第一层时, 使用参数 \texttt{input\_shape}
(整数元组,不包括样本数的轴)。

\textbf{输出尺寸}

与输入相同。


\subsubsection{Masking {\href{https://github.com/keras-team/keras/blob/master/keras/layers/core.py\#L28}{{[}source{]}}}}

\begin{Shaded}
\begin{Highlighting}[]
\NormalTok{keras.layers.Masking(mask_value}\OperatorTok{=}\FloatTok{0.0}\NormalTok{)}
\end{Highlighting}
\end{Shaded}

使用覆盖值覆盖序列,以跳过时间步。

对于输入张量的每一个时间步(张量的第一个维度),
如果所有时间步中输入张量的值与 \texttt{mask\_value} 相等,
那么这个时间步将在所有下游层被覆盖 (跳过) (只要它们支持覆盖)。

如果任何下游层不支持覆盖但仍然收到此类输入覆盖信息,会引发异常。

\textbf{例}

考虑将要喂入一个 LSTM 层的 Numpy 矩阵~\texttt{x}, 尺寸为
\texttt{(samples,\ timesteps,\ features)}。 你想要覆盖时间步 \#3 和
\#5,因为你缺乏这几个 时间步的数据。你可以:

\begin{itemize}
\tightlist
\item
  设置 \texttt{x{[}:,\ 3,\ :{]}\ =\ 0.} 以及
  \texttt{x{[}:,\ 5,\ :{]}\ =\ 0.}
\item
  在 LSTM 层之前,插入一个 \texttt{mask\_value=0} 的~\texttt{Masking}
  层:
\end{itemize}

\begin{Shaded}
\begin{Highlighting}[]
\NormalTok{model }\OperatorTok{=} \NormalTok{Sequential()}
\NormalTok{model.add(Masking(mask_value}\OperatorTok{=}\DecValTok{0}\NormalTok{., input_shape}\OperatorTok{=}\NormalTok{(timesteps, features)))}
\NormalTok{model.add(LSTM(}\DecValTok{32}\NormalTok{))}
\end{Highlighting}
\end{Shaded}
\newpage