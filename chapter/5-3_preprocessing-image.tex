\subsection{图像预处理}
\subsubsection{ImageDataGenerator}\label{imagedatagenerator}

\begin{Shaded}
\begin{Highlighting}[]
\NormalTok{keras.preprocessing.image.ImageDataGenerator(featurewise_center}\OperatorTok{=}\VariableTok{False}\NormalTok{,}
    \NormalTok{samplewise_center}\OperatorTok{=}\VariableTok{False}\NormalTok{,}
    \NormalTok{featurewise_std_normalization}\OperatorTok{=}\VariableTok{False}\NormalTok{,}
    \NormalTok{samplewise_std_normalization}\OperatorTok{=}\VariableTok{False}\NormalTok{,}
    \NormalTok{zca_whitening}\OperatorTok{=}\VariableTok{False}\NormalTok{,}
    \NormalTok{zca_epsilon}\OperatorTok{=}\FloatTok{1e-6}\NormalTok{,}
    \NormalTok{rotation_range}\OperatorTok{=}\DecValTok{0}\NormalTok{.,}
    \NormalTok{width_shift_range}\OperatorTok{=}\DecValTok{0}\NormalTok{.,}
    \NormalTok{height_shift_range}\OperatorTok{=}\DecValTok{0}\NormalTok{.,}
    \NormalTok{shear_range}\OperatorTok{=}\DecValTok{0}\NormalTok{.,}
    \NormalTok{zoom_range}\OperatorTok{=}\DecValTok{0}\NormalTok{.,}
    \NormalTok{channel_shift_range}\OperatorTok{=}\DecValTok{0}\NormalTok{.,}
    \NormalTok{fill_mode}\OperatorTok{=}\StringTok{'nearest'}\NormalTok{,}
    \NormalTok{cval}\OperatorTok{=}\DecValTok{0}\NormalTok{.,}
    \NormalTok{horizontal_flip}\OperatorTok{=}\VariableTok{False}\NormalTok{,}
    \NormalTok{vertical_flip}\OperatorTok{=}\VariableTok{False}\NormalTok{,}
    \NormalTok{rescale}\OperatorTok{=}\VariableTok{None}\NormalTok{,}
    \NormalTok{preprocessing_function}\OperatorTok{=}\VariableTok{None}\NormalTok{,}
    \NormalTok{data_format}\OperatorTok{=}\NormalTok{K.image_data_format())}
\end{Highlighting}
\end{Shaded}

生成批次的带实时数据增益的张量图像数据。数据将按批次无限循环。

\begin{itemize}
\tightlist
\item
  \textbf{参数}:

  \begin{itemize}
  \tightlist
  \item
    \textbf{featurewise\_center}: 布尔值。将输入数据的均值设置为
    0,逐特征进行。
  \item
    \textbf{samplewise\_center}: 布尔值。将每个样本的均值设置为 0。
  \item
    \textbf{featurewise\_std\_normalization}:
    布尔值。将输入除以数据标准差,逐特征进行。
  \item
    \textbf{samplewise\_std\_normalization}:
    布尔值。将每个输入除以其标准差。
  \item
    \textbf{zca\_epsilon}: ZCA 白化的 epsilon 值,默认为 1e-6。
  \item
    \textbf{zca\_whitening}: 布尔值。应用 ZCA 白化。
  \item
    \textbf{rotation\_range}: 整数。随机旋转的度数范围。
  \item
    \textbf{width\_shift\_range}:
    浮点数(总宽度的比例)。随机水平移动的范围。
  \item
    \textbf{height\_shift\_range}:
    浮点数(总高度的比例)。随机垂直移动的范围。
  \item
    \textbf{shear\_range}:
    浮点数。剪切强度(以弧度逆时针方向剪切角度)。
  \item
    \textbf{zoom\_range}: 浮点数 或 {[}lower,
    upper{]}。随机缩放范围。如果是浮点数,\texttt{{[}lower,\ upper{]}\ =\ {[}1-zoom\_range,\ 1+zoom\_range{]}}。
  \item
    \textbf{channel\_shift\_range}: 浮点数。随机通道转换的范围。
  \item
    \textbf{fill\_mode}: \{"constant", "nearest", "reflect" or "wrap"\}
    之一。输入边界以外的点根据给定的模式填充:

    \begin{itemize}
    \tightlist
    \item
      "constant": \texttt{kkkkkkkk\textbar{}abcd\textbar{}kkkkkkkk}
      (\texttt{cval=k})
    \item
      "nearest": \texttt{aaaaaaaa\textbar{}abcd\textbar{}dddddddd}
    \item
      "reflect": \texttt{abcddcba\textbar{}abcd\textbar{}dcbaabcd}
    \item
      "wrap": \texttt{abcdabcd\textbar{}abcd\textbar{}abcdabcd}
    \end{itemize}
  \item
    \textbf{cval}: 浮点数或整数。用于边界之外的点的值,当
    \texttt{fill\_mode\ =\ "constant"} 时。
  \item
    \textbf{horizontal\_flip}: 布尔值。随机水平翻转。
  \item
    \textbf{vertical\_flip}: 布尔值。随机垂直翻转。
  \item
    \textbf{rescale}: 重缩放因子。默认为 None。如果是 None 或
    0,不进行缩放,否则将数据乘以所提供的值(在应用任何其他转换之前)。
  \item
    \textbf{preprocessing\_function}:
    应用于每个输入的函数。这个函数会在任何其他改变之前运行。这个函数需要一个参数:一张图像(秩为
    3 的 Numpy 张量),并且应该输出一个同尺寸的 Numpy 张量。
  \item
    \textbf{data\_format}: \{"channels\_first", "channels\_last"\}
    之一。"channels\_last" 模式表示输入尺寸应该为
    \texttt{(samples,\ height,\ width,\ channels)},"channels\_first"
    模式表示输入尺寸应该为
    \texttt{(samples,\ channels,\ height,\ width)}。默认为 在 Keras
    配置文件 \texttt{\textasciitilde{}/.keras/keras.json} 中的
    \texttt{image\_data\_format} 值。如果你从未设置它,那它就是
    "channels\_last"。
  \end{itemize}
\item
  \textbf{方法}:

  \begin{itemize}
  \tightlist
  \item
    \textbf{fit(x)}:
    根据一组样本数据,计算与数据相关转换有关的内部数据统计信息。当且仅当
    featurewise\_center 或 featurewise\_std\_normalization 或
    zca\_whitening 时才需要。

    \begin{itemize}
    \tightlist
    \item
      \textbf{参数}:

      \begin{itemize}
      \tightlist
      \item
        \textbf{x}: 样本数据。秩应该为
        4。在灰度数据的情况下,通道轴的值应该为 1,在 RGB
        数据的情况下,它应该为 3。
      \item
        \textbf{augment}: 布尔值(默认 False)。是否使用随机样本增益。
      \item
        \textbf{rounds}: 整数(默认 1)。如果
        augment,在数据上进行多少次增益。
      \item
        \textbf{seed}: 整数(默认 None)。随机种子。
      \end{itemize}
    \end{itemize}
  \item
    \textbf{flow(x, y)}: 传入 Numpy 数据和标签数组,生成批次的
    增益的/标准化的 数据。在生成的批次数据上无限制地无限次循环。

    \begin{itemize}
    \tightlist
    \item
      \textbf{参数}:

      \begin{itemize}
      \tightlist
      \item
        \textbf{x}: 数据。秩应该为
        4。在灰度数据的情况下,通道轴的值应该为 1,在 RGB
        数据的情况下,它应该为 3。
      \item
        \textbf{y}: 标签。
      \item
        \textbf{batch\_size}: 整数(默认 32)。
      \item
        \textbf{shuffle}: 布尔值(默认 True)。
      \item
        \textbf{seed}: 整数(默认 None)。
      \item
        \textbf{save\_to\_dir}: None 或 字符串(默认
        None)。这使你可以最佳地指定正在生成的增强图片要保存的目录(用于可视化你在做什么)。
      \item
        \textbf{save\_prefix}: 字符串(默认
        \texttt{\textquotesingle{}\textquotesingle{}})。
        保存图片的文件名前缀(仅当 \texttt{save\_to\_dir} 设置时可用)。
      \item
        \textbf{save\_format}: "png", "jpeg" 之一(仅当
        \texttt{save\_to\_dir} 设置时可用)。默认:"png"。
      \end{itemize}
    \item
      \textbf{yields}: 元组 \texttt{(x,\ y)},其中 \texttt{x}
      是图像数据的 Numpy 数组,\texttt{y} 是相应标签的 Numpy
      数组。生成器将无限循环。
    \end{itemize}
  \item
    \textbf{flow\_from\_directory(directory)}:
    以目录路径为参数,生成批次的 增益的/标准化的
    数据。在生成的批次数据上无限制地无限次循环。

    \begin{itemize}
    \tightlist
    \item
      \textbf{参数}:

      \begin{itemize}
      \tightlist
      \item
        \textbf{directory}:
        目标目录的路径。每个类应该包含一个子目录。任何在子目录下的 PNG,
        JPG, BMP 或 PPM 图像,都将被包含在生成器中。更多细节,详见
        \href{https://gist.github.com/fchollet/0830affa1f7f19fd47b06d4cf89ed44d}{此脚本}。
      \item
        \textbf{target\_size}: 整数元组
        \texttt{(height,\ width)},默认:\texttt{(256,\ 256)}。所有的图像将被调整到的尺寸。
      \item
        \textbf{color\_mode}: "grayscale", "rbg"
        之一。默认:"rgb"。图像是否被转换成1或3个颜色通道。
      \item
        \textbf{classes}: 可选的类的子目录列表(例如
        \texttt{{[}\textquotesingle{}dogs\textquotesingle{},\ \textquotesingle{}cats\textquotesingle{}{]}})。默认:None。如果未提供,类的列表将自动从``目录''下的子目录名称/结构中推断出来,其中每个子目录都将被作为不同的类(类名将按字典序映射到标签的索引)。包含从类名到类索引的映射的字典可以通过\texttt{class\_indices}属性获得。
      \item
        \textbf{class\_mode}: "categorical", "binary", "sparse", "input"
        或 None
        之一。默认:"categorical"。决定返回的标签数组的类型:"categorical"
        将是 2D one-hot 编码标签,"binary" 将是 1D 二进制标签,"sparse"
        将是 1D 整数标签,"input"
        将是与输入图像相同的图像(主要用于与自动编码器一起工作)。如果为
        None,不返回标签(生成器将只产生批量的图像数据,对于
        \texttt{model.predict\_generator()},
        \texttt{model.evaluate\_generator()} 等很有用)。请注意,如果
        class\_mode 为 None,那么数据仍然需要驻留在 \texttt{directory}
        的子目录中才能正常工作。
      \item
        \textbf{batch\_size}: 一批数据的大小(默认 32)。
      \item
        \textbf{shuffle}: 是否混洗数据(默认 True)。
      \item
        \textbf{seed}: 可选随机种子,用于混洗和转换。
      \item
        \textbf{save\_to\_dir}: None 或 字符串(默认
        None)。这使你可以最佳地指定正在生成的增强图片要保存的目录(用于可视化你在做什么)。
      \item
        \textbf{save\_prefix}: 字符串。 保存图片的文件名前缀(仅当
        \texttt{save\_to\_dir} 设置时可用)。
      \item
        \textbf{save\_format}: "png", "jpeg" 之一(仅当
        \texttt{save\_to\_dir} 设置时可用)。默认:"png"。
      \item
        \textbf{follow\_links}: 是否跟踪类子目录下的符号链接(默认
        False)。
      \end{itemize}
    \end{itemize}
  \end{itemize}
\item
  \textbf{例}:
\end{itemize}

使用 \texttt{.flow(x,\ y)} 的例子:

\begin{Shaded}
\begin{Highlighting}[]
\NormalTok{(x_train, y_train), (x_test, y_test) }\OperatorTok{=} \NormalTok{cifar10.load_data()}
\NormalTok{y_train }\OperatorTok{=} \NormalTok{np_utils.to_categorical(y_train, num_classes)}
\NormalTok{y_test }\OperatorTok{=} \NormalTok{np_utils.to_categorical(y_test, num_classes)}

\NormalTok{datagen }\OperatorTok{=} \NormalTok{ImageDataGenerator(}
    \NormalTok{featurewise_center}\OperatorTok{=}\VariableTok{True}\NormalTok{,}
    \NormalTok{featurewise_std_normalization}\OperatorTok{=}\VariableTok{True}\NormalTok{,}
    \NormalTok{rotation_range}\OperatorTok{=}\DecValTok{20}\NormalTok{,}
    \NormalTok{width_shift_range}\OperatorTok{=}\FloatTok{0.2}\NormalTok{,}
    \NormalTok{height_shift_range}\OperatorTok{=}\FloatTok{0.2}\NormalTok{,}
    \NormalTok{horizontal_flip}\OperatorTok{=}\VariableTok{True}\NormalTok{)}

\CommentTok{# 计算特征归一化所需的数量}
\CommentTok{# (如果应用 ZCA 白化,将计算标准差,均值,主成分)}
\NormalTok{datagen.fit(x_train)}

\CommentTok{# 使用实时数据增益的批数据对模型进行拟合:}
\NormalTok{model.fit_generator(datagen.flow(x_train, y_train, batch_size}\OperatorTok{=}\DecValTok{32}\NormalTok{),}
                    \NormalTok{steps_per_epoch}\OperatorTok{=}\BuiltInTok{len}\NormalTok{(x_train) }\OperatorTok{/} \DecValTok{32}\NormalTok{, epochs}\OperatorTok{=}\NormalTok{epochs)}

\CommentTok{# 这里有一个更 「手动」的例子}
\ControlFlowTok{for} \NormalTok{e }\OperatorTok{in} \BuiltInTok{range}\NormalTok{(epochs):}
    \BuiltInTok{print}\NormalTok{(}\StringTok{'Epoch'}\NormalTok{, e)}
    \NormalTok{batches }\OperatorTok{=} \DecValTok{0}
    \ControlFlowTok{for} \NormalTok{x_batch, y_batch }\OperatorTok{in} \NormalTok{datagen.flow(x_train, y_train, batch_size}\OperatorTok{=}\DecValTok{32}\NormalTok{):}
        \NormalTok{model.fit(x_batch, y_batch)}
        \NormalTok{batches }\OperatorTok{+=} \DecValTok{1}
        \ControlFlowTok{if} \NormalTok{batches }\OperatorTok{>=} \BuiltInTok{len}\NormalTok{(x_train) }\OperatorTok{/} \DecValTok{32}\NormalTok{:}
            \CommentTok{# 我们需要手动打破循环,}
            \CommentTok{# 因为生成器会无限循环}
            \ControlFlowTok{break}
\end{Highlighting}
\end{Shaded}

使用 \texttt{.flow\_from\_directory(directory)} 的例子:

\begin{Shaded}
\begin{Highlighting}[]
\NormalTok{train_datagen }\OperatorTok{=} \NormalTok{ImageDataGenerator(}
        \NormalTok{rescale}\OperatorTok{=}\DecValTok{1}\NormalTok{.}\OperatorTok{/}\DecValTok{255}\NormalTok{,}
        \NormalTok{shear_range}\OperatorTok{=}\FloatTok{0.2}\NormalTok{,}
        \NormalTok{zoom_range}\OperatorTok{=}\FloatTok{0.2}\NormalTok{,}
        \NormalTok{horizontal_flip}\OperatorTok{=}\VariableTok{True}\NormalTok{)}

\NormalTok{test_datagen }\OperatorTok{=} \NormalTok{ImageDataGenerator(rescale}\OperatorTok{=}\DecValTok{1}\NormalTok{.}\OperatorTok{/}\DecValTok{255}\NormalTok{)}

\NormalTok{train_generator }\OperatorTok{=} \NormalTok{train_datagen.flow_from_directory(}
        \StringTok{'data/train'}\NormalTok{,}
        \NormalTok{target_size}\OperatorTok{=}\NormalTok{(}\DecValTok{150}\NormalTok{, }\DecValTok{150}\NormalTok{),}
        \NormalTok{batch_size}\OperatorTok{=}\DecValTok{32}\NormalTok{,}
        \NormalTok{class_mode}\OperatorTok{=}\StringTok{'binary'}\NormalTok{)}

\NormalTok{validation_generator }\OperatorTok{=} \NormalTok{test_datagen.flow_from_directory(}
        \StringTok{'data/validation'}\NormalTok{,}
        \NormalTok{target_size}\OperatorTok{=}\NormalTok{(}\DecValTok{150}\NormalTok{, }\DecValTok{150}\NormalTok{),}
        \NormalTok{batch_size}\OperatorTok{=}\DecValTok{32}\NormalTok{,}
        \NormalTok{class_mode}\OperatorTok{=}\StringTok{'binary'}\NormalTok{)}

\NormalTok{model.fit_generator(}
        \NormalTok{train_generator,}
        \NormalTok{steps_per_epoch}\OperatorTok{=}\DecValTok{2000}\NormalTok{,}
        \NormalTok{epochs}\OperatorTok{=}\DecValTok{50}\NormalTok{,}
        \NormalTok{validation_data}\OperatorTok{=}\NormalTok{validation_generator,}
        \NormalTok{validation_steps}\OperatorTok{=}\DecValTok{800}\NormalTok{)}
\end{Highlighting}
\end{Shaded}

同时转换图像和蒙版 (mask) 的例子。

\begin{Shaded}
\begin{Highlighting}[]
\CommentTok{# 创建两个相同参数的实例}
\NormalTok{data_gen_args }\OperatorTok{=} \BuiltInTok{dict}\NormalTok{(featurewise_center}\OperatorTok{=}\VariableTok{True}\NormalTok{,}
                     \NormalTok{featurewise_std_normalization}\OperatorTok{=}\VariableTok{True}\NormalTok{,}
                     \NormalTok{rotation_range}\OperatorTok{=}\DecValTok{90}\NormalTok{.,}
                     \NormalTok{width_shift_range}\OperatorTok{=}\FloatTok{0.1}\NormalTok{,}
                     \NormalTok{height_shift_range}\OperatorTok{=}\FloatTok{0.1}\NormalTok{,}
                     \NormalTok{zoom_range}\OperatorTok{=}\FloatTok{0.2}\NormalTok{)}
\NormalTok{image_datagen }\OperatorTok{=} \NormalTok{ImageDataGenerator(}\OperatorTok{**}\NormalTok{data_gen_args)}
\NormalTok{mask_datagen }\OperatorTok{=} \NormalTok{ImageDataGenerator(}\OperatorTok{**}\NormalTok{data_gen_args)}

\CommentTok{# 为 fit 和 flow 函数提供相同的种子和关键字参数}
\NormalTok{seed }\OperatorTok{=} \DecValTok{1}
\NormalTok{image_datagen.fit(images, augment}\OperatorTok{=}\VariableTok{True}\NormalTok{, seed}\OperatorTok{=}\NormalTok{seed)}
\NormalTok{mask_datagen.fit(masks, augment}\OperatorTok{=}\VariableTok{True}\NormalTok{, seed}\OperatorTok{=}\NormalTok{seed)}

\NormalTok{image_generator }\OperatorTok{=} \NormalTok{image_datagen.flow_from_directory(}
    \StringTok{'data/images'}\NormalTok{,}
    \NormalTok{class_mode}\OperatorTok{=}\VariableTok{None}\NormalTok{,}
    \NormalTok{seed}\OperatorTok{=}\NormalTok{seed)}

\NormalTok{mask_generator }\OperatorTok{=} \NormalTok{mask_datagen.flow_from_directory(}
    \StringTok{'data/masks'}\NormalTok{,}
    \NormalTok{class_mode}\OperatorTok{=}\VariableTok{None}\NormalTok{,}
    \NormalTok{seed}\OperatorTok{=}\NormalTok{seed)}

\CommentTok{# 将生成器组合成一个产生图像和蒙版(mask)的生成器}
\NormalTok{train_generator }\OperatorTok{=} \BuiltInTok{zip}\NormalTok{(image_generator, mask_generator)}

\NormalTok{model.fit_generator(}
    \NormalTok{train_generator,}
    \NormalTok{steps_per_epoch}\OperatorTok{=}\DecValTok{2000}\NormalTok{,}
    \NormalTok{epochs}\OperatorTok{=}\DecValTok{50}\NormalTok{)}
\end{Highlighting}
\end{Shaded}
\newpage
