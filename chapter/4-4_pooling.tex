\subsection{池化层 Pooling}
\subsubsection{MaxPooling1D {\href{https://github.com/keras-team/keras/blob/master/keras/layers/pooling.py\#L57}{{[}source{]}}}}

\begin{Shaded}
\begin{Highlighting}[]
\NormalTok{keras.layers.MaxPooling1D(pool_size}\OperatorTok{=}\DecValTok{2}\NormalTok{, strides}\OperatorTok{=}\VariableTok{None}\NormalTok{, padding}\OperatorTok{=}\StringTok{'valid'}\NormalTok{)}
\end{Highlighting}
\end{Shaded}

对于时序数据的最大池化。

\textbf{参数}

\begin{itemize}
\tightlist
\item
  \textbf{pool\_size}: 整数,最大池化的窗口大小。
\item
  \textbf{strides}: 整数,或者是 \texttt{None}。作为缩小比例的因数。
  例如,2 会使得输入张量缩小一半。 如果是 \texttt{None},那么默认值是
  \texttt{pool\_size}。
\item
  \textbf{padding}: \texttt{"valid"} 或者 \texttt{"same"}
  (区分大小写)。
\end{itemize}

\textbf{输入尺寸}

尺寸是 \texttt{(batch\_size,\ steps,\ features)} 的 3D 张量。

\textbf{输出尺寸}

尺寸是 \texttt{(batch\_size,\ downsampled\_steps,\ features)} 的 3D
张量。




\subsubsection{MaxPooling2D {\href{https://github.com/keras-team/keras/blob/master/keras/layers/pooling.py\#L170}{{[}source{]}}}}

\begin{Shaded}
\begin{Highlighting}[]
\NormalTok{keras.layers.MaxPooling2D(pool_size}\OperatorTok{=}\NormalTok{(}\DecValTok{2}\NormalTok{, }\DecValTok{2}\NormalTok{), strides}\OperatorTok{=}\VariableTok{None}\NormalTok{, padding}\OperatorTok{=}\StringTok{'valid'},\\
\hspace{4cm}\NormalTok{data_format}\OperatorTok{=}\VariableTok{None}\NormalTok{)}
\end{Highlighting}
\end{Shaded}

对于空域数据的最大池化。

\textbf{参数}

\begin{itemize}
\tightlist
\item
  \textbf{pool\_size}: 整数,或者 2
  个整数元组,(垂直方向,水平方向)缩小比例的因数。(2,2)会把输入张量的两个维度都缩小一半。
  如果只使用一个整数,那么两个维度都会使用同样的窗口长度。
\item
  \textbf{strides}: 整数,整数元组或者是 \texttt{None}。 步长值。 如果是
  \texttt{None},那么默认值是 \texttt{pool\_size}。
\item
  \textbf{padding}: \texttt{"valid"} 或者 \texttt{"same"}
  (区分大小写)。
\item
  \textbf{data\_format}: 一个字符串,\texttt{channels\_last}
  (默认值)或者 \texttt{channels\_first}。 输入张量中的维度顺序。
  \texttt{channels\_last} 代表尺寸是
  \texttt{(batch,\ height,\ width,\ channels)} 的输入张量,而
  \texttt{channels\_first} 代表尺寸是
  \texttt{(batch,\ channels,\ height,\ width)} 的输入张量。 默认值根据
  Keras 配置文件 \texttt{\textasciitilde{}/.keras/keras.json} 中的
  \texttt{image\_data\_format} 值来设置。
  如果还没有设置过,那么默认值就是 "channels\_last"。
\end{itemize}

\textbf{输入尺寸}

\begin{itemize}
\tightlist
\item
  如果
  \texttt{data\_format=\textquotesingle{}channels\_last\textquotesingle{}}:
  尺寸是 \texttt{(batch\_size,\ rows,\ cols,\ channels)} 的 4D 张量
\item
  如果
  \texttt{data\_format=\textquotesingle{}channels\_first\textquotesingle{}}:
  尺寸是 \texttt{(batch\_size,\ channels,\ rows,\ cols)} 的 4D 张量
\end{itemize}

\textbf{输出尺寸}

\begin{itemize}
\tightlist
\item
  如果
  \texttt{data\_format=\textquotesingle{}channels\_last\textquotesingle{}}:
  尺寸是 \texttt{(batch\_size,\ pooled\_rows,\ pooled\_cols,\ channels)}
  的 4D 张量
\item
  如果
  \texttt{data\_format=\textquotesingle{}channels\_first\textquotesingle{}}:
  尺寸是 \texttt{(batch\_size,\ channels,\ pooled\_rows,\ pooled\_cols)}
  的 4D 张量
\end{itemize}




\subsubsection{MaxPooling3D {\href{https://github.com/keras-team/keras/blob/master/keras/layers/pooling.py\#L339}{{[}source{]}}}}

\begin{Shaded}
\begin{Highlighting}[]
\NormalTok{keras.layers.MaxPooling3D(pool_size}\OperatorTok{=}\NormalTok{(}\DecValTok{2}\NormalTok{, }\DecValTok{2}\NormalTok{, }\DecValTok{2}\NormalTok{), strides}\OperatorTok{=}\VariableTok{None}\NormalTok{, padding}\OperatorTok{=}\StringTok{'valid'},\\
\hspace{4cm}\NormalTok{data_format}\OperatorTok{=}\VariableTok{None}\NormalTok{)}
\end{Highlighting}
\end{Shaded}

对于 3D(空域,或时空域)数据的最大池化。

\textbf{参数}

\begin{itemize}
\tightlist
\item
  \textbf{pool\_size}: 3 个整数的元组,缩小(维度 1,维度 2,维度
  3)比例的因数。 (2, 2, 2) 会把 3D 输入张量的每个维度缩小一半。
\item
  \textbf{strides}: 3 个整数的元组,或者是 \texttt{None}。步长值。
\item
  \textbf{padding}: \texttt{"valid"} 或者
  \texttt{"same"}(区分大小写)。
\item
  \textbf{data\_format}: 一个字符串,\texttt{channels\_last}
  (默认值)或者 \texttt{channels\_first}。 输入张量中的维度顺序。
  \texttt{channels\_last} 代表尺寸是
  \texttt{(batch,\ spatial\_dim1,\ spatial\_dim2,\ spatial\_dim3,\ channels)}
  的输入张量, 而 \texttt{channels\_first} 代表尺寸是
  \texttt{(batch,\ channels,\ spatial\_dim1,\ spatial\_dim2,\ spatial\_dim3)}
  的输入张量。 默认值根据 Keras 配置文件
  \texttt{\textasciitilde{}/.keras/keras.json} 中的
  \texttt{image\_data\_format} 值来设置。
  如果还没有设置过,那么默认值就是 "channels\_last"。
\end{itemize}

\textbf{输入尺寸}

\begin{itemize}
\tightlist
\item
  如果
  \texttt{data\_format=\textquotesingle{}channels\_last\textquotesingle{}}:
  尺寸是
  \texttt{(batch\_size,\ spatial\_dim1,\ spatial\_dim2,\ spatial\_dim3,\ channels)}
  的 5D 张量
\item
  如果
  \texttt{data\_format=\textquotesingle{}channels\_first\textquotesingle{}}:
  尺寸是
  \texttt{(batch\_size,\ channels,\ spatial\_dim1,\ spatial\_dim2,\ spatial\_dim3)}
  的 5D 张量
\end{itemize}

\textbf{输出尺寸}

\begin{itemize}
\tightlist
\item
  如果
  \texttt{data\_format=\textquotesingle{}channels\_last\textquotesingle{}}:
  尺寸是
  \texttt{(batch\_size,\ pooled\_dim1,\ pooled\_dim2,\ pooled\_dim3,\ channels)}
  的 5D 张量
\item
  如果
  \texttt{data\_format=\textquotesingle{}channels\_first\textquotesingle{}}:
  尺寸是
  \texttt{(batch\_size,\ channels,\ pooled\_dim1,\ pooled\_dim2,\ pooled\_dim3)}
  的 5D 张量
\end{itemize}



\subsubsection{AveragePooling1D  {\href{https://github.com/keras-team/keras/blob/master/keras/layers/pooling.py\#L87}{{[}source{]}}}}

\begin{Shaded}
\begin{Highlighting}[]
\NormalTok{keras.layers.AveragePooling1D(pool_size}\OperatorTok{=}\DecValTok{2}\NormalTok{, strides}\OperatorTok{=}\VariableTok{None}\NormalTok{, padding}\OperatorTok{=}\StringTok{'valid'}\NormalTok{)}
\end{Highlighting}
\end{Shaded}

对于时序数据的平均池化。

\textbf{参数}

\begin{itemize}
\tightlist
\item
  \textbf{pool\_size}: 整数,平均池化的窗口大小。
\item
  \textbf{strides}: 整数,或者是 \texttt{None}。作为缩小比例的因数。
  例如,2 会使得输入张量缩小一半。 如果是 \texttt{None},那么默认值是
  \texttt{pool\_size}。
\item
  \textbf{padding}: \texttt{"valid"} 或者 \texttt{"same"}
  (区分大小写)。
\end{itemize}

\textbf{输入尺寸}

尺寸是 \texttt{(batch\_size,\ steps,\ features)} 的 3D 张量。

\textbf{输出尺寸}

尺寸是 \texttt{(batch\_size,\ downsampled\_steps,\ features)} 的 3D
张量。




\subsubsection{AveragePooling2D {\href{https://github.com/keras-team/keras/blob/master/keras/layers/pooling.py\#L225}{{[}source{]}}}}

\begin{Shaded}
\begin{Highlighting}[]
\NormalTok{keras.layers.AveragePooling2D(pool_size}\OperatorTok{=}\NormalTok{(}\DecValTok{2}\NormalTok{, }\DecValTok{2}\NormalTok{), strides}\OperatorTok{=}\VariableTok{None}\NormalTok{, padding}\OperatorTok{=}\StringTok{'valid'},\\
\hspace{4cm}\NormalTok{data_format}\OperatorTok{=}\VariableTok{None}\NormalTok{)}
\end{Highlighting}
\end{Shaded}

对于空域数据的平均池化。

\textbf{参数}

\begin{itemize}
\tightlist
\item
  \textbf{pool\_size}: 整数,或者 2
  个整数元组,(垂直方向,水平方向)缩小比例的因数。(2,2)会把输入张量的两个维度都缩小一半。
  如果只使用一个整数,那么两个维度都会使用同样的窗口长度。
\item
  \textbf{strides}: 整数,整数元组或者是 \texttt{None}。 步长值。 如果是
  \texttt{None},那么默认值是 \texttt{pool\_size}。
\item
  \textbf{padding}: \texttt{"valid"} 或者 \texttt{"same"}
  (区分大小写)。
\item
  \textbf{data\_format}: 一个字符串,\texttt{channels\_last}
  (默认值)或者 \texttt{channels\_first}。 输入张量中的维度顺序。
  \texttt{channels\_last} 代表尺寸是
  \texttt{(batch,\ height,\ width,\ channels)} 的输入张量,而
  \texttt{channels\_first} 代表尺寸是
  \texttt{(batch,\ channels,\ height,\ width)} 的输入张量。 默认值根据
  Keras 配置文件 \texttt{\textasciitilde{}/.keras/keras.json} 中的
  \texttt{image\_data\_format} 值来设置。
  如果还没有设置过,那么默认值就是 "channels\_last"。
\end{itemize}

\textbf{输入尺寸}

\begin{itemize}
\tightlist
\item
  如果
  \texttt{data\_format=\textquotesingle{}channels\_last\textquotesingle{}}:
  尺寸是 \texttt{(batch\_size,\ rows,\ cols,\ channels)} 的 4D 张量
\item
  如果
  \texttt{data\_format=\textquotesingle{}channels\_first\textquotesingle{}}:
  尺寸是 \texttt{(batch\_size,\ channels,\ rows,\ cols)} 的 4D 张量
\end{itemize}

\textbf{输出尺寸}

\begin{itemize}
\tightlist
\item
  如果
  \texttt{data\_format=\textquotesingle{}channels\_last\textquotesingle{}}:
  尺寸是 \texttt{(batch\_size,\ pooled\_rows,\ pooled\_cols,\ channels)}
  的 4D 张量
\item
  如果
  \texttt{data\_format=\textquotesingle{}channels\_first\textquotesingle{}}:
  尺寸是 \texttt{(batch\_size,\ channels,\ pooled\_rows,\ pooled\_cols)}
  的 4D 张量
\end{itemize}




\subsubsection{AveragePooling3D {\href{https://github.com/keras-team/keras/blob/master/keras/layers/pooling.py\#L389}{{[}source{]}}}}

\begin{Shaded}
\begin{Highlighting}[]
\NormalTok{keras.layers.AveragePooling3D(pool_size}\OperatorTok{=}\NormalTok{(}\DecValTok{2}\NormalTok{, }\DecValTok{2}\NormalTok{, }\DecValTok{2}\NormalTok{), strides}\OperatorTok{=}\VariableTok{None}\NormalTok{, padding}\OperatorTok{=}\StringTok{'valid'},\\
\hspace{4cm}\NormalTok{data_format}\OperatorTok{=}\VariableTok{None}\NormalTok{)}
\end{Highlighting}
\end{Shaded}

对于 3D (空域,或者时空域)数据的平均池化。

\textbf{参数}

\begin{itemize}
\tightlist
\item
  \textbf{pool\_size}: 3 个整数的元组,缩小(维度 1,维度 2,维度
  3)比例的因数。 (2, 2, 2) 会把 3D 输入张量的每个维度缩小一半。
\item
  \textbf{strides}: 3 个整数的元组,或者是 \texttt{None}。步长值。
\item
  \textbf{padding}: \texttt{"valid"} 或者
  \texttt{"same"}(区分大小写)。
\item
  \textbf{data\_format}: 一个字符串,\texttt{channels\_last}
  (默认值)或者 \texttt{channels\_first}。 输入张量中的维度顺序。
  \texttt{channels\_last} 代表尺寸是
  \texttt{(batch,\ spatial\_dim1,\ spatial\_dim2,\ spatial\_dim3,\ channels)}
  的输入张量, 而 \texttt{channels\_first} 代表尺寸是
  \texttt{(batch,\ channels,\ spatial\_dim1,\ spatial\_dim2,\ spatial\_dim3)}
  的输入张量。 默认值根据 Keras 配置文件
  \texttt{\textasciitilde{}/.keras/keras.json} 中的
  \texttt{image\_data\_format} 值来设置。
  如果还没有设置过,那么默认值就是 "channels\_last"。
\end{itemize}

\textbf{输入尺寸}

\begin{itemize}
\tightlist
\item
  如果
  \texttt{data\_format=\textquotesingle{}channels\_last\textquotesingle{}}:
  尺寸是
  \texttt{(batch\_size,\ spatial\_dim1,\ spatial\_dim2,\ spatial\_dim3,\ channels)}
  的 5D 张量
\item
  如果
  \texttt{data\_format=\textquotesingle{}channels\_first\textquotesingle{}}:
  尺寸是
  \texttt{(batch\_size,\ channels,\ spatial\_dim1,\ spatial\_dim2,\ spatial\_dim3)}
  的 5D 张量
\end{itemize}

\textbf{输出尺寸}

\begin{itemize}
\tightlist
\item
  如果
  \texttt{data\_format=\textquotesingle{}channels\_last\textquotesingle{}}:
  尺寸是
  \texttt{(batch\_size,\ pooled\_dim1,\ pooled\_dim2,\ pooled\_dim3,\ channels)}
  的 5D 张量
\item
  如果
  \texttt{data\_format=\textquotesingle{}channels\_first\textquotesingle{}}:
  尺寸是
  \texttt{(batch\_size,\ channels,\ pooled\_dim1,\ pooled\_dim2,\ pooled\_dim3)}
  的 5D 张量
\end{itemize}




\subsubsection{GlobalMaxPooling1D {\href{https://github.com/keras-team/keras/blob/master/keras/layers/pooling.py\#L470}{{[}source{]}}}}

\begin{Shaded}
\begin{Highlighting}[]
\NormalTok{keras.layers.GlobalMaxPooling1D()}
\end{Highlighting}
\end{Shaded}

对于时序数据的全局最大池化。

\textbf{输入尺寸}

尺寸是 \texttt{(batch\_size,\ steps,\ features)} 的 3D 张量。

\textbf{输出尺寸}

尺寸是 \texttt{(batch\_size,\ features)} 的 2D 张量。




\subsubsection{GlobalAveragePooling1D {\href{https://github.com/keras-team/keras/blob/master/keras/layers/pooling.py\#L455}{{[}source{]}}}}

\begin{Shaded}
\begin{Highlighting}[]
\NormalTok{keras.layers.GlobalAveragePooling1D()}
\end{Highlighting}
\end{Shaded}

对于时序数据的全局平均池化。

\textbf{输入尺寸}

尺寸是 \texttt{(batch\_size,\ steps,\ features)} 的 3D 张量。

\textbf{输出尺寸}

尺寸是 \texttt{(batch\_size,\ features)} 的 2D 张量。




\subsubsection{GlobalMaxPooling2D {\href{https://github.com/keras-team/keras/blob/master/keras/layers/pooling.py\#L545}{{[}source{]}}}}

\begin{Shaded}
\begin{Highlighting}[]
\NormalTok{keras.layers.GlobalMaxPooling2D(data_format}\OperatorTok{=}\VariableTok{None}\NormalTok{)}
\end{Highlighting}
\end{Shaded}

对于空域数据的全局最大池化。

\textbf{参数}

\begin{itemize}
\tightlist
\item
  \textbf{data\_format}: 一个字符串,\texttt{channels\_last}
  (默认值)或者 \texttt{channels\_first}。 输入张量中的维度顺序。
  \texttt{channels\_last} 代表尺寸是
  \texttt{(batch,\ height,\ width,\ channels)} 的输入张量,而
  \texttt{channels\_first} 代表尺寸是
  \texttt{(batch,\ channels,\ height,\ width)} 的输入张量。 默认值根据
  Keras 配置文件 \texttt{\textasciitilde{}/.keras/keras.json} 中的
  \texttt{image\_data\_format} 值来设置。
  如果还没有设置过,那么默认值就是 "channels\_last"。
\end{itemize}

\textbf{输入尺寸}

\begin{itemize}
\tightlist
\item
  如果
  \texttt{data\_format=\textquotesingle{}channels\_last\textquotesingle{}}:
  尺寸是 \texttt{(batch\_size,\ rows,\ cols,\ channels)} 的 4D 张量
\item
  如果
  \texttt{data\_format=\textquotesingle{}channels\_first\textquotesingle{}}:
  尺寸是 \texttt{(batch\_size,\ channels,\ rows,\ cols)} 的 4D 张量
\end{itemize}

\textbf{输出尺寸}

尺寸是 \texttt{(batch\_size,\ channels)} 的 2D 张量


\subsubsection{GlobalAveragePooling2D {\href{https://github.com/keras-team/keras/blob/master/keras/layers/pooling.py\#L510}{{[}source{]}}}}

\begin{Shaded}
\begin{Highlighting}[]
\NormalTok{keras.layers.GlobalAveragePooling2D(data_format}\OperatorTok{=}\VariableTok{None}\NormalTok{)}
\end{Highlighting}
\end{Shaded}

对于空域数据的全局平均池化。

\textbf{参数}

\begin{itemize}
\tightlist
\item
  \textbf{data\_format}: 一个字符串,\texttt{channels\_last}
  (默认值)或者 \texttt{channels\_first}。 输入张量中的维度顺序。
  \texttt{channels\_last} 代表尺寸是
  \texttt{(batch,\ height,\ width,\ channels)} 的输入张量,而
  \texttt{channels\_first} 代表尺寸是
  \texttt{(batch,\ channels,\ height,\ width)} 的输入张量。 默认值根据
  Keras 配置文件 \texttt{\textasciitilde{}/.keras/keras.json} 中的
  \texttt{image\_data\_format} 值来设置。
  如果还没有设置过,那么默认值就是 "channels\_last"。
\end{itemize}

\textbf{输入尺寸}

\begin{itemize}
\tightlist
\item
  如果
  \texttt{data\_format=\textquotesingle{}channels\_last\textquotesingle{}}:
  尺寸是 \texttt{(batch\_size,\ rows,\ cols,\ channels)} 的 4D 张量
\item
  如果
  \texttt{data\_format=\textquotesingle{}channels\_first\textquotesingle{}}:
  尺寸是 \texttt{(batch\_size,\ channels,\ rows,\ cols)} 的 4D 张量
\end{itemize}

\textbf{输出尺寸}

尺寸是 \texttt{(batch\_size,\ channels)} 的 2D 张量




\subsubsection{GlobalMaxPooling3D {\href{https://github.com/keras-team/keras/blob/master/keras/layers/pooling.py\#L639}{{[}source{]}}}}

\begin{Shaded}
\begin{Highlighting}[]
\NormalTok{keras.layers.GlobalMaxPooling3D(data_format}\OperatorTok{=}\VariableTok{None}\NormalTok{)}
\end{Highlighting}
\end{Shaded}

对于 3D 数据的全局最大池化。

\textbf{参数}

\begin{itemize}
\tightlist
\item
  \textbf{data\_format}: 一个字符串,\texttt{channels\_last}
  (默认值)或者 \texttt{channels\_first}。 输入张量中的维度顺序。
  \texttt{channels\_last} 代表尺寸是
  \texttt{(batch,\ spatial\_dim1,\ spatial\_dim2,\ spatial\_dim3,\ channels)}
  的输入张量,而 \texttt{channels\_first} 代表尺寸是
  \texttt{(batch,\ channels,\ spatial\_dim1,\ spatial\_dim2,\ spatial\_dim3)}
  的输入张量。 默认值根据 Keras 配置文件
  \texttt{\textasciitilde{}/.keras/keras.json} 中的
  \texttt{image\_data\_format} 值来设置。
  如果还没有设置过,那么默认值就是 ``channels\_last''。
\end{itemize}

\textbf{输入尺寸}

\begin{itemize}
\tightlist
\item
  如果
  \texttt{data\_format=\textquotesingle{}channels\_last\textquotesingle{}}:
  尺寸是
  \texttt{(batch\_size,\ spatial\_dim1,\ spatial\_dim2,\ spatial\_dim3,\ channels)}
  的 5D 张量
\item
  如果
  \texttt{data\_format=\textquotesingle{}channels\_first\textquotesingle{}}:
  尺寸是
  \texttt{(batch\_size,\ channels,\ spatial\_dim1,\ spatial\_dim2,\ spatial\_dim3)}
  的 5D 张量
\end{itemize}

\textbf{输出尺寸}

尺寸是 \texttt{(batch\_size,\ channels)} 的 2D 张量


\subsubsection{GlobalAveragePooling3D {\href{https://github.com/keras-team/keras/blob/master/keras/layers/pooling.py\#L604}{{[}source{]}}}}

\begin{Shaded}
\begin{Highlighting}[]
\NormalTok{keras.layers.GlobalAveragePooling3D(data_format}\OperatorTok{=}\VariableTok{None}\NormalTok{)}
\end{Highlighting}
\end{Shaded}

对于 3D 数据的全局平均池化。

\textbf{参数}

\begin{itemize}
\tightlist
\item
  \textbf{data\_format}: 一个字符串,\texttt{channels\_last}
  (默认值)或者 \texttt{channels\_first}。 输入张量中的维度顺序。
  \texttt{channels\_last} 代表尺寸是
  \texttt{(batch,\ spatial\_dim1,\ spatial\_dim2,\ spatial\_dim3,\ channels)}
  的输入张量,而 \texttt{channels\_first} 代表尺寸是
  \texttt{(batch,\ channels,\ spatial\_dim1,\ spatial\_dim2,\ spatial\_dim3)}
  的输入张量。 默认值根据 Keras 配置文件
  \texttt{\textasciitilde{}/.keras/keras.json} 中的
  \texttt{image\_data\_format} 值来设置。
  如果还没有设置过,那么默认值就是 ``channels\_last''。
\end{itemize}

\textbf{输入尺寸}

\begin{itemize}
\tightlist
\item
  如果
  \texttt{data\_format=\textquotesingle{}channels\_last\textquotesingle{}}:
  尺寸是
  \texttt{(batch\_size,\ spatial\_dim1,\ spatial\_dim2,\ spatial\_dim3,\ channels)}
  的 5D 张量
\item
  如果
  \texttt{data\_format=\textquotesingle{}channels\_first\textquotesingle{}}:
  尺寸是
  \texttt{(batch\_size,\ channels,\ spatial\_dim1,\ spatial\_dim2,\ spatial\_dim3)}
  的 5D 张量
\end{itemize}

\textbf{输出尺寸}

尺寸是 \texttt{(batch\_size,\ channels)} 的 2D 张量

\newpage
