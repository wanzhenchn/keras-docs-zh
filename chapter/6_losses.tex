\section{损失函数 Losses}\label{losses}
    
\subsection{损失函数的使用}

损失函数(或称目标函数、优化评分函数)是编译模型时所需的两个参数之一:

\begin{Shaded}
\begin{Highlighting}[]
\NormalTok{model.}\BuiltInTok{compile}\NormalTok{(loss}\OperatorTok{=}\StringTok{'mean_squared_error'}\NormalTok{, optimizer}\OperatorTok{=}\StringTok{'sgd'}\NormalTok{)}
\end{Highlighting}
\end{Shaded}

\begin{Shaded}
\begin{Highlighting}[]
\ImportTok{from} \NormalTok{keras }\ImportTok{import} \NormalTok{losses}

\NormalTok{model.}\BuiltInTok{compile}\NormalTok{(loss}\OperatorTok{=}\NormalTok{losses.mean_squared_error, optimizer}\OperatorTok{=}\StringTok{'sgd'}\NormalTok{)}
\end{Highlighting}
\end{Shaded}

你可以传递一个现有的损失函数名,或者一个TensorFlow/Theano符号函数。
该符号函数为每个数据点返回一个标量,有以下两个参数:

\begin{itemize}
\tightlist
\item
  \textbf{y\_true}: 真实标签. TensorFlow/Theano张量。
\item
  \textbf{y\_pred}: 预测值.
  TensorFlow/Theano张量,其shape与y\_true相同。
\end{itemize}

实际的优化目标是所有数据点的输出数组的平均值。

有关这些函数的几个例子,请查看\href{https://github.com/keras-team/keras/blob/master/keras/losses.py}{losses
source}。

\subsection{可用损失函数}\label{ux53efux7528ux635fux5931ux51fdux6570}

\subsubsection{mean\_squared\_error}\label{meanux5fsquaredux5ferror}

\begin{Shaded}
\begin{Highlighting}[]
\NormalTok{mean_squared_error(y_true, y_pred)}
\end{Highlighting}
\end{Shaded}



\subsubsection{mean\_absolute\_error}\label{meanux5fabsoluteux5ferror}

\begin{Shaded}
\begin{Highlighting}[]
\NormalTok{mean_absolute_error(y_true, y_pred)}
\end{Highlighting}
\end{Shaded}



\subsubsection{mean\_absolute\_percentage\_error}\label{meanux5fabsoluteux5fpercentageux5ferror}

\begin{Shaded}
\begin{Highlighting}[]
\NormalTok{mean_absolute_percentage_error(y_true, y_pred)}
\end{Highlighting}
\end{Shaded}



\subsubsection{mean\_squared\_logarithmic\_error}\label{meanux5fsquaredux5flogarithmicux5ferror}

\begin{Shaded}
\begin{Highlighting}[]
\NormalTok{mean_squared_logarithmic_error(y_true, y_pred)}
\end{Highlighting}
\end{Shaded}



\subsubsection{squared\_hinge}\label{squaredux5fhinge}

\begin{Shaded}
\begin{Highlighting}[]
\NormalTok{squared_hinge(y_true, y_pred)}
\end{Highlighting}
\end{Shaded}



\subsubsection{hinge}\label{hinge}

\begin{Shaded}
\begin{Highlighting}[]
\NormalTok{hinge(y_true, y_pred)}
\end{Highlighting}
\end{Shaded}



\subsubsection{categorical\_hinge}\label{categoricalux5fhinge}

\begin{Shaded}
\begin{Highlighting}[]
\NormalTok{categorical_hinge(y_true, y_pred)}
\end{Highlighting}
\end{Shaded}



\subsubsection{logcosh}\label{logcosh}

\begin{Shaded}
\begin{Highlighting}[]
\NormalTok{logcosh(y_true, y_pred)}
\end{Highlighting}
\end{Shaded}

预测误差的双曲余弦的对数。

对于小的\texttt{x},\texttt{log(cosh(x))}近似等于\texttt{(x\ **\ 2)\ /\ 2}。对于大的\texttt{x},近似于\texttt{abs(x)\ -\ log(2)}。这表示'logcosh'与均方误差大致相同,但是不会受到偶尔疯狂的错误预测的强烈影响。

\textbf{Arguments}

\begin{itemize}
\tightlist
\item
  \textbf{y\_true}: 目标真实值的张量。
\item
  \textbf{y\_pred}: 目标预测值的张量。
\end{itemize}

\textbf{Returns}

每个样本都有一个标量损失的张量。



\subsubsection{categorical\_crossentropy}\label{categoricalux5fcrossentropy}

\begin{Shaded}
\begin{Highlighting}[]
\NormalTok{categorical_crossentropy(y_true, y_pred)}
\end{Highlighting}
\end{Shaded}



\subsubsection{sparse\_categorical\_crossentropy}\label{sparseux5fcategoricalux5fcrossentropy}

\begin{Shaded}
\begin{Highlighting}[]
\NormalTok{sparse_categorical_crossentropy(y_true, y_pred)}
\end{Highlighting}
\end{Shaded}



\subsubsection{binary\_crossentropy}\label{binaryux5fcrossentropy}

\begin{Shaded}
\begin{Highlighting}[]
\NormalTok{binary_crossentropy(y_true, y_pred)}
\end{Highlighting}
\end{Shaded}



\subsubsection{kullback\_leibler\_divergence}\label{kullbackux5fleiblerux5fdivergence}

\begin{Shaded}
\begin{Highlighting}[]
\NormalTok{kullback_leibler_divergence(y_true, y_pred)}
\end{Highlighting}
\end{Shaded}



\subsubsection{poisson}\label{poisson}

\begin{Shaded}
\begin{Highlighting}[]
\NormalTok{poisson(y_true, y_pred)}
\end{Highlighting}
\end{Shaded}



\subsubsection{cosine\_proximity}\label{cosineux5fproximity}

\begin{Shaded}
\begin{Highlighting}[]
\NormalTok{cosine_proximity(y_true, y_pred)}
\end{Highlighting}
\end{Shaded}



\textbf{注意}:
当使用\texttt{categorical\_crossentropy}损失时,你的目标值应该是分类格式
(即,如果你有10个类,每个样本的目标值应该是一个10维的向量,这个向量除了表示类别的那个索引为1,其他均为0)。
为了将 \emph{整数目标值} 转换为
\emph{分类目标值},你可以使用Keras实用函数\texttt{to\_categorical}:

\begin{Shaded}
\begin{Highlighting}[]
\ImportTok{from} \NormalTok{keras.utils.np_utils }\ImportTok{import} \NormalTok{to_categorical}

\NormalTok{categorical_labels }\OperatorTok{=} \NormalTok{to_categorical(int_labels, num_classes}\OperatorTok{=}\VariableTok{None}\NormalTok{)}
\end{Highlighting}
\end{Shaded}
\newpage