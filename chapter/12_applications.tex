
\section{预训练模型 Applications}\label{applications}

Keras的应用模块(keras.applications)提供了带有预训练权值的深度学习模型,这些模型可以用来进行预测、特征提取和微调(fine-tuning)。

当你初始化一个预训练模型时,会自动下载权值到
\texttt{\textasciitilde{}/.keras/models/} 目录下。

\subsection{可用的模型}\label{ux53efux7528ux7684ux6a21ux578b}

\textbf{在 ImageNet
上预训练过的用于图像分类的模型:}\label{ux5728-imagenet-ux4e0aux9884ux8badux7ec3ux8fc7ux7684ux7528ux4e8eux56feux50cfux5206ux7c7bux7684ux6a21ux578b}

\begin{itemize}
\tightlist
\item
  \hyperref[xception]{Xception}
\item
  \hyperref[vgg16]{VGG16}
\item
  \hyperref[vgg19]{VGG19}
\item
  \hyperref[resnet50]{EesNet50}
\item
  \hyperref[inceptionv3]{InceptionV3}
\item
  \hyperref[inceptionresnetv2]{InceptionResNetV2}
\item
  \hyperref[mobilenet]{MobileNet}
\item
  \hyperref[densenet]{DenseNet}
\item
  \hyperref[nasnet]{NASnet}
\end{itemize}

所有的这些模型(除了 Xception 和 MobileNet
外)都兼容Theano和Tensorflow,并会自动按照位于
\texttt{\textasciitilde{}/.keras/keras.json}
的配置文件中设置的图像数据格式来构建模型。举个例子,如果你设置
\texttt{image\_data\_format=channels\_last},则加载的模型将按照
TensorFlow
的维度顺序来构造,即``高度-宽度-深度''(Height-Width-Depth)的顺序。

Xception 模型仅适用于 TensorFlow,因为它依赖于 SeparableConvolution
层。MobileNet 模型仅适用于 TensorFlow,因为它依赖于 DepthwiseConvolution
层。



\subsection{图像分类模型的示例代码}\label{ux56feux50cfux5206ux7c7bux6a21ux578bux7684ux793aux4f8bux4ee3ux7801}

\subsubsection{使用 ResNet50 进行 ImageNet
分类}\label{ux4f7fux7528-resnet50-ux8fdbux884c-imagenet-ux5206ux7c7b}

\begin{Shaded}
\begin{Highlighting}[]
\ImportTok{from} \NormalTok{keras.applications.resnet50 }\ImportTok{import} \NormalTok{ResNet50}
\ImportTok{from} \NormalTok{keras.preprocessing }\ImportTok{import} \NormalTok{image}
\ImportTok{from} \NormalTok{keras.applications.resnet50 }\ImportTok{import} \NormalTok{preprocess_input, decode_predictions}
\ImportTok{import} \NormalTok{numpy }\ImportTok{as} \NormalTok{np}

\NormalTok{model }\OperatorTok{=} \NormalTok{ResNet50(weights}\OperatorTok{=}\StringTok{'imagenet'}\NormalTok{)}

\NormalTok{img_path }\OperatorTok{=} \StringTok{'elephant.jpg'}
\NormalTok{img }\OperatorTok{=} \NormalTok{image.load_img(img_path, target_size}\OperatorTok{=}\NormalTok{(}\DecValTok{224}\NormalTok{, }\DecValTok{224}\NormalTok{))}
\NormalTok{x }\OperatorTok{=} \NormalTok{image.img_to_array(img)}
\NormalTok{x }\OperatorTok{=} \NormalTok{np.expand_dims(x, axis}\OperatorTok{=}\DecValTok{0}\NormalTok{)}
\NormalTok{x }\OperatorTok{=} \NormalTok{preprocess_input(x)}

\NormalTok{preds }\OperatorTok{=} \NormalTok{model.predict(x)}
\CommentTok{# decode the results into a list of tuples (class, description, probability)}
\CommentTok{# (one such list for each sample in the batch)}
\BuiltInTok{print}\NormalTok{(}\StringTok{'Predicted:'}\NormalTok{, decode_predictions(preds, top}\OperatorTok{=}\DecValTok{3}\NormalTok{)[}\DecValTok{0}\NormalTok{])}
\CommentTok{# Predicted: [(u'n02504013', u'Indian_elephant', 0.82658225), (u'n01871265', u'tusker', 0.1122357), (u'n02504458', u'African_elephant', 0.061040461)]}
\end{Highlighting}
\end{Shaded}

\subsubsection{使用 VGG16
提取特征}\label{ux4f7fux7528-vgg16-ux63d0ux53d6ux7279ux5f81}

\begin{Shaded}
\begin{Highlighting}[]
\ImportTok{from} \NormalTok{keras.applications.vgg16 }\ImportTok{import} \NormalTok{VGG16}
\ImportTok{from} \NormalTok{keras.preprocessing }\ImportTok{import} \NormalTok{image}
\ImportTok{from} \NormalTok{keras.applications.vgg16 }\ImportTok{import} \NormalTok{preprocess_input}
\ImportTok{import} \NormalTok{numpy }\ImportTok{as} \NormalTok{np}

\NormalTok{model }\OperatorTok{=} \NormalTok{VGG16(weights}\OperatorTok{=}\StringTok{'imagenet'}\NormalTok{, include_top}\OperatorTok{=}\VariableTok{False}\NormalTok{)}

\NormalTok{img_path }\OperatorTok{=} \StringTok{'elephant.jpg'}
\NormalTok{img }\OperatorTok{=} \NormalTok{image.load_img(img_path, target_size}\OperatorTok{=}\NormalTok{(}\DecValTok{224}\NormalTok{, }\DecValTok{224}\NormalTok{))}
\NormalTok{x }\OperatorTok{=} \NormalTok{image.img_to_array(img)}
\NormalTok{x }\OperatorTok{=} \NormalTok{np.expand_dims(x, axis}\OperatorTok{=}\DecValTok{0}\NormalTok{)}
\NormalTok{x }\OperatorTok{=} \NormalTok{preprocess_input(x)}

\NormalTok{features }\OperatorTok{=} \NormalTok{model.predict(x)}
\end{Highlighting}
\end{Shaded}

\subsubsection{从VGG19的任意中间层中抽取特征}\label{ux4ecevgg19ux7684ux4efbux610fux4e2dux95f4ux5c42ux4e2dux62bdux53d6ux7279ux5f81}

\begin{Shaded}
\begin{Highlighting}[]
\ImportTok{from} \NormalTok{keras.applications.vgg19 }\ImportTok{import} \NormalTok{VGG19}
\ImportTok{from} \NormalTok{keras.preprocessing }\ImportTok{import} \NormalTok{image}
\ImportTok{from} \NormalTok{keras.applications.vgg19 }\ImportTok{import} \NormalTok{preprocess_input}
\ImportTok{from} \NormalTok{keras.models }\ImportTok{import} \NormalTok{Model}
\ImportTok{import} \NormalTok{numpy }\ImportTok{as} \NormalTok{np}

\NormalTok{base_model }\OperatorTok{=} \NormalTok{VGG19(weights}\OperatorTok{=}\StringTok{'imagenet'}\NormalTok{)}
\NormalTok{model }\OperatorTok{=} \NormalTok{Model(inputs}\OperatorTok{=}\NormalTok{base_model.}\BuiltInTok{input}\NormalTok{, outputs}\OperatorTok{=}\NormalTok{base_model.get_layer(}\StringTok{'block4_pool'}\NormalTok{).output)}

\NormalTok{img_path }\OperatorTok{=} \StringTok{'elephant.jpg'}
\NormalTok{img }\OperatorTok{=} \NormalTok{image.load_img(img_path, target_size}\OperatorTok{=}\NormalTok{(}\DecValTok{224}\NormalTok{, }\DecValTok{224}\NormalTok{))}
\NormalTok{x }\OperatorTok{=} \NormalTok{image.img_to_array(img)}
\NormalTok{x }\OperatorTok{=} \NormalTok{np.expand_dims(x, axis}\OperatorTok{=}\DecValTok{0}\NormalTok{)}
\NormalTok{x }\OperatorTok{=} \NormalTok{preprocess_input(x)}

\NormalTok{block4_pool_features }\OperatorTok{=} \NormalTok{model.predict(x)}
\end{Highlighting}
\end{Shaded}

\subsubsection{在新类上微调
InceptionV3}\label{ux5728ux65b0ux7c7bux4e0aux5faeux8c03-inceptionv3}

\begin{Shaded}
\begin{Highlighting}[]
\ImportTok{from} \NormalTok{keras.applications.inception_v3 }\ImportTok{import} \NormalTok{InceptionV3}
\ImportTok{from} \NormalTok{keras.preprocessing }\ImportTok{import} \NormalTok{image}
\ImportTok{from} \NormalTok{keras.models }\ImportTok{import} \NormalTok{Model}
\ImportTok{from} \NormalTok{keras.layers }\ImportTok{import} \NormalTok{Dense, GlobalAveragePooling2D}
\ImportTok{from} \NormalTok{keras }\ImportTok{import} \NormalTok{backend }\ImportTok{as} \NormalTok{K}

\CommentTok{# 构建不带分类器的预训练模型}
\NormalTok{base_model }\OperatorTok{=} \NormalTok{InceptionV3(weights}\OperatorTok{=}\StringTok{'imagenet'}\NormalTok{, include_top}\OperatorTok{=}\VariableTok{False}\NormalTok{)}

\CommentTok{# 添加全局平均池化层}
\NormalTok{x }\OperatorTok{=} \NormalTok{base_model.output}
\NormalTok{x }\OperatorTok{=} \NormalTok{GlobalAveragePooling2D()(x)}

\CommentTok{# 添加一个全连接层}
\NormalTok{x }\OperatorTok{=} \NormalTok{Dense(}\DecValTok{1024}\NormalTok{, activation}\OperatorTok{=}\StringTok{'relu'}\NormalTok{)(x)}

\CommentTok{# 添加一个分类器,假设我们有200个类}
\NormalTok{predictions }\OperatorTok{=} \NormalTok{Dense(}\DecValTok{200}\NormalTok{, activation}\OperatorTok{=}\StringTok{'softmax'}\NormalTok{)(x)}

\CommentTok{# 构建我们需要训练的完整模型}
\NormalTok{model }\OperatorTok{=} \NormalTok{Model(inputs}\OperatorTok{=}\NormalTok{base_model.}\BuiltInTok{input}\NormalTok{, outputs}\OperatorTok{=}\NormalTok{predictions)}

\CommentTok{# 首先,我们只训练顶部的几层(随机初始化的层)}
\CommentTok{# 锁住所有 InceptionV3 的卷积层}
\ControlFlowTok{for} \NormalTok{layer }\OperatorTok{in} \NormalTok{base_model.layers:}
    \NormalTok{layer.trainable }\OperatorTok{=} \VariableTok{False}

\CommentTok{# 编译模型(一定要在锁层以后操作)}
\NormalTok{model.}\BuiltInTok{compile}\NormalTok{(optimizer}\OperatorTok{=}\StringTok{'rmsprop'}\NormalTok{, loss}\OperatorTok{=}\StringTok{'categorical_crossentropy'}\NormalTok{)}

\CommentTok{# 在新的数据集上训练几代}
\NormalTok{model.fit_generator(...)}

\CommentTok{# 现在顶层应该训练好了,让我们开始微调 Inception V3 的卷积层。}
\CommentTok{# 我们会锁住底下的几层,然后训练其余的顶层。}

\CommentTok{# 让我们看看每一层的名字和层号,看看我们应该锁多少层呢:}
\ControlFlowTok{for} \NormalTok{i, layer }\OperatorTok{in} \BuiltInTok{enumerate}\NormalTok{(base_model.layers):}
   \BuiltInTok{print}\NormalTok{(i, layer.name)}

\CommentTok{# 我们选择训练最上面的两个 Inception block}
\CommentTok{# 也就是说锁住前面249层,然后放开之后的层。}
\ControlFlowTok{for} \NormalTok{layer }\OperatorTok{in} \NormalTok{model.layers[:}\DecValTok{249}\NormalTok{]:}
   \NormalTok{layer.trainable }\OperatorTok{=} \VariableTok{False}
\ControlFlowTok{for} \NormalTok{layer }\OperatorTok{in} \NormalTok{model.layers[}\DecValTok{249}\NormalTok{:]:}
   \NormalTok{layer.trainable }\OperatorTok{=} \VariableTok{True}

\CommentTok{# 我们需要重新编译模型,才能使上面的修改生效}
\CommentTok{# 让我们设置一个很低的学习率,使用 SGD 来微调}
\ImportTok{from} \NormalTok{keras.optimizers }\ImportTok{import} \NormalTok{SGD}
\NormalTok{model.}\BuiltInTok{compile}\NormalTok{(optimizer}\OperatorTok{=}\NormalTok{SGD(lr}\OperatorTok{=}\FloatTok{0.0001}\NormalTok{, momentum}\OperatorTok{=}\FloatTok{0.9}\NormalTok{), loss}\OperatorTok{=}\StringTok{'categorical_crossentropy'}\NormalTok{)}

\CommentTok{# 我们继续训练模型,这次我们训练最后两个 Inception block}
\CommentTok{# 和两个全连接层}
\NormalTok{model.fit_generator(...)}
\end{Highlighting}
\end{Shaded}

\subsubsection{通过自定义输入 tensor 构建
InceptionV3}\label{ux901aux8fc7ux81eaux5b9aux4e49ux8f93ux5165-tensor-ux6784ux5efa-inceptionv3}

\begin{Shaded}
\begin{Highlighting}[]
\ImportTok{from} \NormalTok{keras.applications.inception_v3 }\ImportTok{import} \NormalTok{InceptionV3}
\ImportTok{from} \NormalTok{keras.layers }\ImportTok{import} \NormalTok{Input}

\CommentTok{# this could also be the output a different Keras model or layer}
\NormalTok{input_tensor }\OperatorTok{=} \NormalTok{Input(shape}\OperatorTok{=}\NormalTok{(}\DecValTok{224}\NormalTok{, }\DecValTok{224}\NormalTok{, }\DecValTok{3}\NormalTok{))  }\CommentTok{# this assumes K.image_data_format() == 'channels_last'}

\NormalTok{model }\OperatorTok{=} \NormalTok{InceptionV3(input_tensor}\OperatorTok{=}\NormalTok{input_tensor, weights}\OperatorTok{=}\StringTok{'imagenet'}\NormalTok{, include_top}\OperatorTok{=}\VariableTok{True}\NormalTok{)}
\end{Highlighting}
\end{Shaded}


\newpage
\subsection{模型概览}

\begin{longtable}[c]{cccccc}
\toprule
模型 & 大小 & Top-1 准确率 & Top-5 准确率 & 参数数量 &
深度\tabularnewline
\midrule
\endhead
\hyperref[xception]{Xception} & 88 MB & 0.790 & 0.945 & 22,910,480 &
126\tabularnewline
\hyperref[vgg16]{VGG16} & 528 MB & 0.715 & 0.901 & 138,357,544 &
23\tabularnewline
\hyperref[vgg19]{VGG19} & 549 MB & 0.727 & 0.910 & 143,667,240 &
26\tabularnewline
\hyperref[resnet50]{EesNet50} & 99 MB & 0.759 & 0.929 & 25,636,712 &
168\tabularnewline
\hyperref[inceptionv3]{InceptionV3} & 92 MB & 0.788 & 0.944 & 23,851,784 &
159\tabularnewline
\hyperref[inceptionresnetv2]{InceptionResNetV2} & 215 MB & 0.804 & 0.953 & 55,873,736 &
572\tabularnewline
\hyperref[mobilenet]{MobileNet} & 17 MB & 0.665 & 0.871 & 4,253,864 &
88\tabularnewline
\hyperref[densenet]{DenseNet121} & 33 MB & 0.745 & 0.918 & 8,062,504 &
121\tabularnewline
\hyperref[densenet]{DenseNet169} & 57 MB & 0.759 & 0.928 & 14,307,880 &
169\tabularnewline
\hyperref[densenet]{DenseNet201} & 80 MB & 0.770 & 0.933 & 20,242,984 &
201\tabularnewline
\bottomrule
\end{longtable}

Top-1 准确率和 Top-5 准确率都是在 ImageNet 验证集上的结果。



\subsubsection{Xception}\label{xception}

\begin{Shaded}
\begin{Highlighting}[]
\NormalTok{keras.applications.xception.Xception(include_top}\OperatorTok{=}\VariableTok{True}\NormalTok{, weights}\OperatorTok{=}\StringTok{'imagenet'}, \\
\hspace{3cm}\NormalTok{input_tensor}\OperatorTok{=}\VariableTok{None}\NormalTok{, input_shape}\OperatorTok{=}\VariableTok{None}\NormalTok{, pooling}\OperatorTok{=}\VariableTok{None}\NormalTok{, classes}\OperatorTok{=}\DecValTok{1000}\NormalTok{)}
\end{Highlighting}
\end{Shaded}

在 ImageNet 上预训练的 Xception V1 模型。

在 ImageNet 上,该模型取得了验证集 top1 0.790 和 top5 0.945 的准确率。

注意,该模型目前仅能在 TensorFlow 后端使用,因为它依赖
\texttt{SeparableConvolution} 层,目前该层只支持 \texttt{channels\_last}
的维度顺序(高度、宽度、通道)。

模型默认输入尺寸是 299x299。

\textbf{参数}

\begin{itemize}
\tightlist
\item
  include\_top: 是否包括顶层的全连接层。
\item
  weights: \texttt{None} 代表随机初始化,
  \texttt{\textquotesingle{}imagenet\textquotesingle{}} 代表加载在
  ImageNet 上预训练的权值。
\item
  input\_tensor: 可选,Keras tensor 作为模型的输入(比如
  \texttt{layers.Input()} 输出的 tensor)
\item
  input\_shape:
  可选,输入尺寸元组,仅当\texttt{include\_top=False}时有效(不然输入形状必须是
  \texttt{(299,\ 299,\ 3)},因为预训练模型是以这个大小训练的)。输入尺寸必须是三个数字,且宽高必须不小于
  71,比如 \texttt{(150,\ 150,\ 3)} 是一个合法的输入尺寸。
\item
  pooling: 可选,当 \texttt{include\_top} 为 \texttt{False}
  时,该参数指定了特征提取时的池化方式。

  \begin{itemize}
  \tightlist
  \item
    \texttt{None}
    代表不池化,直接输出最后一层卷积层的输出,该输出是一个四维张量。
  \item
    \texttt{\textquotesingle{}avg\textquotesingle{}}
    代表全局平均池化(GLobalAveragePool2D),相当于在最后一层卷积层后面再加一层全局平均池化层,输出是一个二维张量。
  \item
    \texttt{\textquotesingle{}max\textquotesingle{}} 代表全局最大池化
  \end{itemize}
\item
  classes: 可选,图片分类的类别数,仅当 \texttt{include\_top} 为
  \texttt{True} 并且不加载预训练权值时可用。
\end{itemize}

\textbf{返回值}

一个 Keras \texttt{Model} 对象.

\textbf{参考文献}\label{ux53c2ux8003ux6587ux732e}

\begin{itemize}
\tightlist
\item
  \href{https://arxiv.org/abs/1610.02357}{Xception: Deep Learning with
  Depthwise Separable Convolutions}
\end{itemize}

\textbf{License}\label{license}

预训练权值由我们自己训练而来,基于 MIT license 发布。



\subsubsection{VGG16}\label{vgg16}

\begin{Shaded}
\begin{Highlighting}[]
\NormalTok{keras.applications.vgg16.VGG16(include_top}\OperatorTok{=}\VariableTok{True}\NormalTok{, weights}\OperatorTok{=}\StringTok{'imagenet'}, \\
\hspace{3cm}\NormalTok{input_tensor}\OperatorTok{=}\VariableTok{None}\NormalTok{, input_shape}\OperatorTok{=}\VariableTok{None}\NormalTok{, pooling}\OperatorTok{=}\VariableTok{None}\NormalTok{, classes}\OperatorTok{=}\DecValTok{1000}\NormalTok{)}
\end{Highlighting}
\end{Shaded}

VGG16 模型,权值由 ImageNet 训练而来。

该模型在 \texttt{Theano} 和 \texttt{TensorFlow} 后端均可使用,并接受
\texttt{channels\_first} 和 \texttt{channels\_last}
两种输入维度顺序(高度,宽度,通道)。

模型默认输入尺寸是 224x224。

\textbf{参数}\label{ux53c2ux6570-1}

\begin{itemize}
\tightlist
\item
  include\_top: 是否包括顶层的全连接层。
\item
  weights: \texttt{None} 代表随机初始化,
  \texttt{\textquotesingle{}imagenet\textquotesingle{}} 代表加载在
  ImageNet 上预训练的权值。
\item
  input\_tensor: 可选,Keras tensor 作为模型的输入(比如
  \texttt{layers.Input()} 输出的 tensor)
\item
  input\_shape: 可选,输入尺寸元组,仅当 \texttt{include\_top=False}
  时有效(不然输入形状必须是 \texttt{(224,\ 224,\ 3)}
  (\texttt{channels\_last} 格式)或 \texttt{(3,\ 224,\ 224)}
  (\texttt{channels\_first}
  格式),因为预训练模型是以这个大小训练的)。输入尺寸必须是三个数字,且宽高必须不小于
  48,比如 \texttt{(200,\ 200,\ 3)} 是一个合法的输入尺寸。
\item
  pooling: 可选,当 \texttt{include\_top} 为 \texttt{False}
  时,该参数指定了特征提取时的池化方式。

  \begin{itemize}
  \tightlist
  \item
    \texttt{None}
    代表不池化,直接输出最后一层卷积层的输出,该输出是一个四维张量。
  \item
    \texttt{\textquotesingle{}avg\textquotesingle{}}
    代表全局平均池化(GLobalAveragePool2D),相当于在最后一层卷积层后面再加一层全局平均池化层,输出是一个二维张量。
  \item
    \texttt{\textquotesingle{}max\textquotesingle{}} 代表全局最大池化
  \end{itemize}
\item
  classes: 可选,图片分类的类别数,仅当 \texttt{include\_top} 为
  \texttt{True} 并且不加载预训练权值时可用。
\end{itemize}

\textbf{返回值}\label{ux8fd4ux56deux503c-1}

一个 Keras \texttt{Model} 对象。

\textbf{参考文献}\label{ux53c2ux8003ux6587ux732e-1}

\begin{itemize}
\tightlist
\item
  \href{https://arxiv.org/abs/1409.1556}{Very Deep Convolutional
  Networks for Large-Scale Image
  Recognition}:如果在研究中使用了VGG,请引用该论文。
\end{itemize}

\textbf{License}\label{license-1}

预训练权值由
\href{http://www.robots.ox.ac.uk/~vgg/research/very_deep/}{VGG at
Oxford} 发布的预训练权值移植而来,基于
\href{https://creativecommons.org/licenses/by/4.0/}{Creative Commons
Attribution License}。



\subsubsection{VGG19}\label{vgg19}

\begin{Shaded}
\begin{Highlighting}[]
\NormalTok{keras.applications.vgg19.VGG19(include_top}\OperatorTok{=}\VariableTok{True}\NormalTok{, weights}\OperatorTok{=}\StringTok{'imagenet'}, \\
\hspace{3cm}\NormalTok{input_tensor}\OperatorTok{=}\VariableTok{None}\NormalTok{, input_shape}\OperatorTok{=}\VariableTok{None}\NormalTok{, pooling}\OperatorTok{=}\VariableTok{None}\NormalTok{, classes}\OperatorTok{=}\DecValTok{1000}\NormalTok{)}
\end{Highlighting}
\end{Shaded}

VGG19 模型,权值由 ImageNet 训练而来。

该模型在 \texttt{Theano} 和 \texttt{TensorFlow} 后端均可使用,并接受
\texttt{channels\_first} 和 \texttt{channels\_last}
两种输入维度顺序(高度,宽度,通道)。

模型默认输入尺寸是 224x224。

\textbf{参数}\label{ux53c2ux6570-2}

\begin{itemize}
\tightlist
\item
  include\_top: 是否包括顶层的全连接层。
\item
  weights: \texttt{None} 代表随机初始化,
  \texttt{\textquotesingle{}imagenet\textquotesingle{}} 代表加载在
  ImageNet 上预训练的权值。
\item
  input\_tensor: 可选,Keras tensor 作为模型的输入(比如
  \texttt{layers.Input()} 输出的 tensor)
\item
  input\_shape: 可选,输入尺寸元组,仅当 \texttt{include\_top=False}
  时有效(不然输入形状必须是 \texttt{(224,\ 224,\ 3)}
  (\texttt{channels\_last} 格式)或 \texttt{(3,\ 224,\ 224)}
  (\texttt{channels\_first}
  格式),因为预训练模型是以这个大小训练的)。输入尺寸必须是三个数字,且宽高必须不小于
  48,比如 \texttt{(200,\ 200,\ 3)} 是一个合法的输入尺寸。
\item
  pooling: 可选,当 \texttt{include\_top} 为 \texttt{False}
  时,该参数指定了特征提取时的池化方式。

  \begin{itemize}
  \tightlist
  \item
    \texttt{None}
    代表不池化,直接输出最后一层卷积层的输出,该输出是一个四维张量。
  \item
    \texttt{\textquotesingle{}avg\textquotesingle{}}
    代表全局平均池化(GLobalAveragePool2D),相当于在最后一层卷积层后面再加一层全局平均池化层,输出是一个二维张量。
  \item
    \texttt{\textquotesingle{}max\textquotesingle{}} 代表全局最大池化
  \end{itemize}
\item
  classes: 可选,图片分类的类别数,仅当 \texttt{include\_top} 为
  \texttt{True} 并且不加载预训练权值时可用。
\end{itemize}

\textbf{返回值}\label{ux8fd4ux56deux503c-2}

一个 Keras \texttt{Model} 对象。

\textbf{参考文献}\label{ux53c2ux8003ux6587ux732e-2}

\begin{itemize}
\tightlist
\item
  \href{https://arxiv.org/abs/1409.1556}{Very Deep Convolutional
  Networks for Large-Scale Image
  Recognition}:如果在研究中使用了VGG,请引用该论文。
\end{itemize}

\textbf{License}\label{license-2}

预训练权值由
\href{http://www.robots.ox.ac.uk/~vgg/research/very_deep/}{VGG at
Oxford} 发布的预训练权值移植而来,基于
\href{https://creativecommons.org/licenses/by/4.0/}{Creative Commons
Attribution License}。



\subsubsection{ResNet50}\label{resnet50}

\begin{Shaded}
\begin{Highlighting}[]
\NormalTok{keras.applications.resnet50.ResNet50(include_top}\OperatorTok{=}\VariableTok{True}\NormalTok{, weights}\OperatorTok{=}\StringTok{'imagenet'}, \\
\hspace{3cm}\NormalTok{input_tensor}\OperatorTok{=}\VariableTok{None}\NormalTok{, input_shape}\OperatorTok{=}\VariableTok{None}\NormalTok{, pooling}\OperatorTok{=}\VariableTok{None}\NormalTok{, classes}\OperatorTok{=}\DecValTok{1000}\NormalTok{)}
\end{Highlighting}
\end{Shaded}

ResNet50 模型,权值由 ImageNet 训练而来。

该模型在 \texttt{Theano} 和 \texttt{TensorFlow} 后端均可使用,并接受
\texttt{channels\_first} 和 \texttt{channels\_last}
两种输入维度顺序(高度,宽度,通道)。

模型默认输入尺寸是 224x224。

\textbf{参数}\label{ux53c2ux6570-3}

\begin{itemize}
\tightlist
\item
  include\_top: 是否包括顶层的全连接层。
\item
  weights: \texttt{None} 代表随机初始化,
  \texttt{\textquotesingle{}imagenet\textquotesingle{}} 代表加载在
  ImageNet 上预训练的权值。
\item
  input\_tensor: 可选,Keras tensor 作为模型的输入(比如
  \texttt{layers.Input()} 输出的 tensor)
\item
  input\_shape: 可选,输入尺寸元组,仅当 \texttt{include\_top=False}
  时有效(不然输入形状必须是 \texttt{(224,\ 224,\ 3)}
  (\texttt{channels\_last} 格式)或 \texttt{(3,\ 224,\ 224)}
  (\texttt{channels\_first}
  格式),因为预训练模型是以这个大小训练的)。输入尺寸必须是三个数字,且宽高必须不小于
  197,比如 \texttt{(200,\ 200,\ 3)} 是一个合法的输入尺寸。
\item
  pooling: 可选,当 \texttt{include\_top} 为 \texttt{False}
  时,该参数指定了特征提取时的池化方式。

  \begin{itemize}
  \tightlist
  \item
    \texttt{None}
    代表不池化,直接输出最后一层卷积层的输出,该输出是一个四维张量。
  \item
    \texttt{\textquotesingle{}avg\textquotesingle{}}
    代表全局平均池化(GLobalAveragePool2D),相当于在最后一层卷积层后面再加一层全局平均池化层,输出是一个二维张量。
  \item
    \texttt{\textquotesingle{}max\textquotesingle{}} 代表全局最大池化
  \end{itemize}
\item
  classes: 可选,图片分类的类别数,仅当 \texttt{include\_top} 为
  \texttt{True} 并且不加载预训练权值时可用。
\end{itemize}

\textbf{返回值}\label{ux8fd4ux56deux503c-3}

一个 Keras \texttt{Model} 对象。

\textbf{参考文献}\label{ux53c2ux8003ux6587ux732e-3}

\begin{itemize}
\tightlist
\item
  \href{https://arxiv.org/abs/1512.03385}{Deep Residual Learning for
  Image Recognition}
\end{itemize}

\textbf{License}\label{license-3}

预训练权值由
\href{https://github.com/KaimingHe/deep-residual-networks}{Kaiming He}
发布的预训练权值移植而来,基于
\href{https://github.com/KaimingHe/deep-residual-networks/blob/master/LICENSE}{MIT
license}。



\subsubsection{InceptionV3}\label{inceptionv3}

\begin{Shaded}
\begin{Highlighting}[]
\NormalTok{keras.applications.inception_v3.InceptionV3(include_top}\OperatorTok{=}\VariableTok{True}\NormalTok{, weights}\OperatorTok{=}\StringTok{'imagenet'}, \\
\hspace{3cm}\NormalTok{input_tensor}\OperatorTok{=}\VariableTok{None}\NormalTok{, input_shape}\OperatorTok{=}\VariableTok{None}\NormalTok{, pooling}\OperatorTok{=}\VariableTok{None}\NormalTok{, classes}\OperatorTok{=}\DecValTok{1000}\NormalTok{)}
\end{Highlighting}
\end{Shaded}

Inception V3 模型,权值由 ImageNet 训练而来。

该模型在 \texttt{Theano} 和 \texttt{TensorFlow} 后端均可使用,并接受
\texttt{channels\_first} 和 \texttt{channels\_last}
两种输入维度顺序(高度,宽度,通道)。

模型默认输入尺寸是 299x299。

\textbf{参数}\label{ux53c2ux6570-4}

\begin{itemize}
\tightlist
\item
  include\_top: 是否包括顶层的全连接层。
\item
  weights: \texttt{None} 代表随机初始化,
  \texttt{\textquotesingle{}imagenet\textquotesingle{}} 代表加载在
  ImageNet 上预训练的权值。
\item
  input\_tensor: 可选,Keras tensor 作为模型的输入(比如
  \texttt{layers.Input()} 输出的 tensor)
\item
  input\_shape: 可选,输入尺寸元组,仅当 \texttt{include\_top=False}
  时有效(不然输入形状必须是 \texttt{(299,\ 299,\ 3)}
  (\texttt{channels\_last} 格式)或 \texttt{(3,\ 299,\ 299)}
  (\texttt{channels\_first}
  格式),因为预训练模型是以这个大小训练的)。输入尺寸必须是三个数字,且宽高必须不小于
  139,比如 \texttt{(150,\ 150,\ 3)} 是一个合法的输入尺寸。
\item
  pooling: 可选,当 \texttt{include\_top} 为 \texttt{False}
  时,该参数指定了特征提取时的池化方式。

  \begin{itemize}
  \tightlist
  \item
    \texttt{None}
    代表不池化,直接输出最后一层卷积层的输出,该输出是一个四维张量。
  \item
    \texttt{\textquotesingle{}avg\textquotesingle{}}
    代表全局平均池化(GLobalAveragePool2D),相当于在最后一层卷积层后面再加一层全局平均池化层,输出是一个二维张量。
  \item
    \texttt{\textquotesingle{}max\textquotesingle{}} 代表全局最大池化
  \end{itemize}
\item
  classes: 可选,图片分类的类别数,仅当 \texttt{include\_top} 为
  \texttt{True} 并且不加载预训练权值时可用。
\end{itemize}

\textbf{返回值}\label{ux8fd4ux56deux503c-4}

一个 Keras \texttt{Model} 对象。

\textbf{参考文献}\label{ux53c2ux8003ux6587ux732e-4}

\begin{itemize}
\tightlist
\item
  \href{http://arxiv.org/abs/1512.00567}{Rethinking the Inception
  Architecture for Computer Vision}
\end{itemize}

\textbf{License}\label{license-4}

预训练权值基于
\href{https://github.com/tensorflow/models/blob/master/LICENSE}{Apache
License}。



\subsubsection{InceptionResNetV2}\label{inceptionresnetv2}

\begin{Shaded}
\begin{Highlighting}[]
\NormalTok{keras.applications.inception_resnet_v2.InceptionResNetV2(include_top}\OperatorTok{=}\VariableTok{True}\NormalTok{, weights}\OperatorTok{=}\StringTok{'imagenet'}\NormalTok{, input_tensor}\OperatorTok{=}\VariableTok{None}, \\
\hspace{3cm}\NormalTok{input_shape}\OperatorTok{=}\VariableTok{None}\NormalTok{, pooling}\OperatorTok{=}\VariableTok{None}\NormalTok{, classes}\OperatorTok{=}\DecValTok{1000}\NormalTok{)}
\end{Highlighting}
\end{Shaded}

Inception-ResNet V2 模型,权值由 ImageNet 训练而来。

该模型在 \texttt{Theano} 和 \texttt{TensorFlow} 后端均可使用,并接受
\texttt{channels\_first} 和 \texttt{channels\_last}
两种输入维度顺序(高度,宽度,通道)。

模型默认输入尺寸是 299x299。

\textbf{参数}\label{ux53c2ux6570-5}

\begin{itemize}
\tightlist
\item
  include\_top: 是否包括顶层的全连接层。
\item
  weights: \texttt{None} 代表随机初始化,
  \texttt{\textquotesingle{}imagenet\textquotesingle{}} 代表加载在
  ImageNet 上预训练的权值。
\item
  input\_tensor: 可选,Keras tensor 作为模型的输入(比如
  \texttt{layers.Input()} 输出的 tensor)
\item
  input\_shape: 可选,输入尺寸元组,仅当 \texttt{include\_top=False}
  时有效(不然输入形状必须是 \texttt{(299,\ 299,\ 3)}
  (\texttt{channels\_last} 格式)或 \texttt{(3,\ 299,\ 299)}
  (\texttt{channels\_first}
  格式),因为预训练模型是以这个大小训练的)。输入尺寸必须是三个数字,且宽高必须不小于
  139,比如 \texttt{(150,\ 150,\ 3)} 是一个合法的输入尺寸。
\item
  pooling: 可选,当 \texttt{include\_top} 为 \texttt{False}
  时,该参数指定了特征提取时的池化方式。

  \begin{itemize}
  \tightlist
  \item
    \texttt{None}
    代表不池化,直接输出最后一层卷积层的输出,该输出是一个四维张量。
  \item
    \texttt{\textquotesingle{}avg\textquotesingle{}}
    代表全局平均池化(GLobalAveragePool2D),相当于在最后一层卷积层后面再加一层全局平均池化层,输出是一个二维张量。
  \item
    \texttt{\textquotesingle{}max\textquotesingle{}} 代表全局最大池化
  \end{itemize}
\item
  classes: 可选,图片分类的类别数,仅当 \texttt{include\_top} 为
  \texttt{True} 并且不加载预训练权值时可用。
\end{itemize}

\textbf{返回值}\label{ux8fd4ux56deux503c-5}

一个 Keras \texttt{Model} 对象。

\textbf{参考文献}\label{ux53c2ux8003ux6587ux732e-5}

\begin{itemize}
\tightlist
\item
  \href{http://arxiv.org/abs/1512.00567}{Rethinking the Inception
  Architecture for Computer Vision}
\end{itemize}

\textbf{License}\label{license-5}

预训练权值基于
\href{https://github.com/tensorflow/models/blob/master/LICENSE}{Apache
License}。



\subsubsection{MobileNet}\label{mobilenet}

\begin{Shaded}
\begin{Highlighting}[]
\NormalTok{keras.applications.mobilenet.MobileNet(input_shape}\OperatorTok{=}\VariableTok{None}\NormalTok{, alpha}\OperatorTok{=}\FloatTok{1.0}, \\
\hspace{3cm}\NormalTok{depth_multiplier}\OperatorTok{=}\DecValTok{1}\NormalTok{, dropout}\OperatorTok{=}\FloatTok{1e-3}\NormalTok{, include_top}\OperatorTok{=}\VariableTok{True}, \\
\hspace{3cm}\NormalTok{weights}\OperatorTok{=}\StringTok{'imagenet'}\NormalTok{, input_tensor}\OperatorTok{=}\VariableTok{None}\NormalTok{, pooling}\OperatorTok{=}\VariableTok{None}\NormalTok{, classes}\OperatorTok{=}\DecValTok{1000}\NormalTok{)}
\end{Highlighting}
\end{Shaded}

在 ImageNet 上预训练的 MobileNet 模型。

注意,该模型目前仅能在 TensorFlow 后端使用,因为它依赖
\texttt{SeparableConvolution} 层,目前该层只支持 \texttt{channels\_last}
的维度顺序(高度、宽度、通道)。

要通过 \texttt{load\_model} 载入 MobileNet 模型,你需要导入自定义对象
\texttt{relu6} 和 \texttt{DepthwiseConv2D} 并通过
\texttt{custom\_objects} 传参。

下面是示例代码:

\begin{Shaded}
\begin{Highlighting}[]
\NormalTok{model }\OperatorTok{=} \NormalTok{load_model(}\StringTok{'mobilenet.h5'}\NormalTok{, custom_objects}\OperatorTok{=}\NormalTok{\{}
                   \StringTok{'relu6'}\NormalTok{: mobilenet.relu6,}
                   \CommentTok{'DepthwiseConv2D'}\NormalTok{: mobilenet.DepthwiseConv2D\})}
\end{Highlighting}
\end{Shaded}

模型默认输入尺寸是 224x224.

\textbf{参数}\label{ux53c2ux6570-6}

\begin{itemize}
\tightlist
\item
  input\_shape: 可选,输入尺寸元组,仅当 \texttt{include\_top=False}
  时有效(不然输入形状必须是 \texttt{(224,\ 224,\ 3)}
  ,因为预训练模型是以这个大小训练的)。输入尺寸必须是三个数字,且宽高必须不小于
  32,比如 \texttt{(200,\ 200,\ 3)} 是一个合法的输入尺寸。
\item
  alpha: 控制网络的宽度:

  \begin{itemize}
  \tightlist
  \item
    如果 \texttt{alpha} \textless{} 1.0,则同比例减少每层的滤波器个数。
  \item
    如果 \texttt{alpha} \textgreater{}
    1.0,则同比例增加每层的滤波器个数。
  \item
    如果 \texttt{alpha} = 1,使用论文默认的滤波器个数
  \end{itemize}
\item
  depth\_multiplier: depthwise卷积的深度乘子,也称为(分辨率乘子)
\item
  dropout: dropout 概率
\item
  include\_top: 是否包括顶层的全连接层。
\item
  weights: \texttt{None} 代表随机初始化,
  \texttt{\textquotesingle{}imagenet\textquotesingle{}} 代表加载在
  ImageNet 上预训练的权值。
\item
  input\_tensor: 可选,Keras tensor 作为模型的输入(比如
  \texttt{layers.Input()} 输出的 tensor)
\item
  pooling: 可选,当 \texttt{include\_top} 为 \texttt{False}
  时,该参数指定了特征提取时的池化方式。

  \begin{itemize}
  \tightlist
  \item
    \texttt{None}
    代表不池化,直接输出最后一层卷积层的输出,该输出是一个四维张量。
  \item
    \texttt{\textquotesingle{}avg\textquotesingle{}}
    代表全局平均池化(GLobalAveragePool2D),相当于在最后一层卷积层后面再加一层全局平均池化层,输出是一个二维张量。
  \item
    \texttt{\textquotesingle{}max\textquotesingle{}} 代表全局最大池化
  \end{itemize}
\item
  classes: 可选,图片分类的类别数,仅当 \texttt{include\_top} 为
  \texttt{True} 并且不加载预训练权值时可用。
\end{itemize}

\textbf{返回}\label{ux8fd4ux56de}

一个 Keras \texttt{Model} 对象。

\textbf{参考文献}\label{ux53c2ux8003ux6587ux732e-6}

\begin{itemize}
\tightlist
\item
  \href{https://arxiv.org/pdf/1704.04861.pdf}{MobileNets: Efficient
  Convolutional Neural Networks for Mobile Vision Applications}
\end{itemize}

\textbf{License}\label{license-6}

预训练权值基于
\href{https://github.com/tensorflow/models/blob/master/LICENSE}{Apache
License}。



\subsubsection{DenseNet}\label{densenet}

\begin{Shaded}
\begin{Highlighting}[]
\NormalTok{keras.applications.densenet.DenseNet121(include_top}\OperatorTok{=}\VariableTok{True}\NormalTok{, weights}\OperatorTok{=}\StringTok{'imagenet'}, \\
\hspace{3cm}\NormalTok{input_tensor}\OperatorTok{=}\VariableTok{None}\NormalTok{, input_shape}\OperatorTok{=}\VariableTok{None}\NormalTok{, pooling}\OperatorTok{=}\VariableTok{None}\NormalTok{, classes}\OperatorTok{=}\DecValTok{1000}\NormalTok{)}
\NormalTok{keras.applications.densenet.DenseNet169(include_top}\OperatorTok{=}\VariableTok{True}\NormalTok{, weights}\OperatorTok{=}\StringTok{'imagenet'}, \\
\hspace{3cm}\NormalTok{input_tensor}\OperatorTok{=}\VariableTok{None}\NormalTok{, input_shape}\OperatorTok{=}\VariableTok{None}\NormalTok{, pooling}\OperatorTok{=}\VariableTok{None}\NormalTok{, classes}\OperatorTok{=}\DecValTok{1000}\NormalTok{)}
\NormalTok{keras.applications.densenet.DenseNet201(include_top}\OperatorTok{=}\VariableTok{True}\NormalTok{, weights}\OperatorTok{=}\StringTok{'imagenet'}, \\
\hspace{3cm}\NormalTok{input_tensor}\OperatorTok{=}\VariableTok{None}\NormalTok{, input_shape}\OperatorTok{=}\VariableTok{None}\NormalTok{, pooling}\OperatorTok{=}\VariableTok{None}\NormalTok{, classes}\OperatorTok{=}\DecValTok{1000}\NormalTok{)}
\end{Highlighting}
\end{Shaded}

可以选择载入在 ImageNet 上的预训练权值。如果你在使用 TensorFlow
为了发挥最佳性能,请在 \texttt{\textasciitilde{}/.keras/keras.json} 的
Keras 配置文件中设置
\texttt{image\_data\_format=\textquotesingle{}channels\_last\textquotesingle{}}。

模型和权值兼容 TensorFlow、Theano 和 CNTK。可以在你的 Keras
配置文件中指定数据格式。

\textbf{参数}\label{ux53c2ux6570-7}

\begin{itemize}
\tightlist
\item
  blocks: 四个 Dense Layers 的 block 数量。
\item
  include\_top: 是否包括顶层的全连接层。
\item
  weights: \texttt{None} 代表随机初始化,
  \texttt{\textquotesingle{}imagenet\textquotesingle{}} 代表加载在
  ImageNet 上预训练的权值。
\item
  input\_tensor: 可选,Keras tensor 作为模型的输入(比如
  \texttt{layers.Input()} 输出的 tensor)
\item
  input\_shape: 可选,输入尺寸元组,仅当 \texttt{include\_top=False}
  时有效(不然输入形状必须是 \texttt{(224,\ 224,\ 3)}
  (\texttt{channels\_last} 格式)或 \texttt{(3,\ 224,\ 224)}
  (\texttt{channels\_first}
  格式),因为预训练模型是以这个大小训练的)。输入尺寸必须是三个数字。
\item
  pooling: 可选,当 \texttt{include\_top} 为 \texttt{False}
  时,该参数指定了特征提取时的池化方式。

  \begin{itemize}
  \tightlist
  \item
    \texttt{None}
    代表不池化,直接输出最后一层卷积层的输出,该输出是一个四维张量。
  \item
    \texttt{\textquotesingle{}avg\textquotesingle{}}
    代表全局平均池化(GLobalAveragePool2D),相当于在最后一层卷积层后面再加一层全局平均池化层,输出是一个二维张量。
  \item
    \texttt{\textquotesingle{}max\textquotesingle{}} 代表全局最大池化
  \end{itemize}
\item
  classes: 可选,图片分类的类别数,仅当 \texttt{include\_top} 为
  \texttt{True} 并且不加载预训练权值时可用。
\end{itemize}


A Keras model instance.

\textbf{返回}\label{ux8fd4ux56de-1}

一个 Keras \texttt{Model} 对象。

\textbf{参考文献}\label{ux53c2ux8003ux6587ux732e-7}

预训练权值基于
\href{https://github.com/liuzhuang13/DenseNet/blob/master/LICENSE}{BSD
3-clause License}。



\subsubsection{NASNet}\label{nasnet}

\begin{Shaded}
\begin{Highlighting}[]
\NormalTok{keras.applications.nasnet.NASNetLarge(input_shape}\OperatorTok{=}\VariableTok{None}\NormalTok{, include_top}\OperatorTok{=}\VariableTok{True}, \\
\hspace{3cm}\NormalTok{weights}\OperatorTok{=}\StringTok{'imagenet'}\NormalTok{, input_tensor}\OperatorTok{=}\VariableTok{None}\NormalTok{, pooling}\OperatorTok{=}\VariableTok{None}\NormalTok{, classes}\OperatorTok{=}\DecValTok{1000}\NormalTok{)}
\NormalTok{keras.applications.nasnet.NASNetMobile(input_shape}\OperatorTok{=}\VariableTok{None}\NormalTok{, include_top}\OperatorTok{=}\VariableTok{True}, \\
\hspace{3cm}\NormalTok{weights}\OperatorTok{=}\StringTok{'imagenet'}\NormalTok{, input_tensor}\OperatorTok{=}\VariableTok{None}\NormalTok{, pooling}\OperatorTok{=}\VariableTok{None}\NormalTok{, classes}\OperatorTok{=}\DecValTok{1000}\NormalTok{)}
\end{Highlighting}
\end{Shaded}

在 ImageNet 上预训练的神经结构搜索网络模型(NASNet)。

注意,该模型目前仅能在 TensorFlow 后端使用,因此它只支持
\texttt{channels\_last} 的维度顺序(高度、宽度、通道),可以在
\texttt{\textasciitilde{}/.keras/keras.json} Keras 配置文件中设置。

NASNetLarge 默认的输入尺寸是 331x331,NASNetMobile 默认的输入尺寸是
224x224。

\textbf{参数}\label{ux53c2ux6570-8}

\begin{itemize}
\tightlist
\item
  input\_shape: 可选,输入尺寸元组,仅当 \texttt{include\_top=False}
  时有效(不然对于 NASNetMobile 模型来说,输入形状必须是
  \texttt{(224,\ 224,\ 3)} (\texttt{channels\_last} 格式),对于
  NASNetLarge 来说,输入形状必须是 \texttt{(331,\ 331,\ 3)}
  (\texttt{channels\_last} 格式)。输入尺寸必须是三个数字。
\item
  include\_top: 是否包括顶层的全连接层。
\item
  weights: \texttt{None} 代表随机初始化,
  \texttt{\textquotesingle{}imagenet\textquotesingle{}} 代表加载在
  ImageNet 上预训练的权值。
\item
  input\_tensor: 可选,Keras tensor 作为模型的输入(比如
  \texttt{layers.Input()} 输出的 tensor)
\item
  input\_shape: 可选,输入尺寸元组,仅当 \texttt{include\_top=False}
  时有效(不然输入形状必须是 \texttt{(224,\ 224,\ 3)}
  (\texttt{channels\_last} 格式)或 \texttt{(3,\ 224,\ 224)}
  (\texttt{channels\_first}
  格式),因为预训练模型是以这个大小训练的)。输入尺寸必须是三个数字。
\item
  pooling: 可选,当 \texttt{include\_top} 为 \texttt{False}
  时,该参数指定了特征提取时的池化方式。

  \begin{itemize}
  \tightlist
  \item
    \texttt{None}
    代表不池化,直接输出最后一层卷积层的输出,该输出是一个四维张量。
  \item
    \texttt{\textquotesingle{}avg\textquotesingle{}}
    代表全局平均池化(GLobalAveragePool2D),相当于在最后一层卷积层后面再加一层全局平均池化层,输出是一个二维张量。
  \item
    \texttt{\textquotesingle{}max\textquotesingle{}} 代表全局最大池化
  \end{itemize}
\item
  classes: 可选,图片分类的类别数,仅当 \texttt{include\_top} 为
  \texttt{True} 并且不加载预训练权值时可用。
\end{itemize}

\textbf{返回}\label{ux8fd4ux56de-2}

一个 Keras \texttt{Model} 对象。

\textbf{参考文献}\label{ux53c2ux8003ux6587ux732e-8}

\begin{itemize}
\tightlist
\item
  \href{https://arxiv.org/abs/1707.07012}{Learning Transferable
  Architectures for Scalable Image Recognition}
\end{itemize}

\textbf{License}\label{license-7}

预训练权值基于
\href{https://github.com/tensorflow/models/blob/master/LICENSE}{Apache
License}。

\newpage
