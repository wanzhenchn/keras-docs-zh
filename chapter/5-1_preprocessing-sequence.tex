\section{数据预处理}
\subsection{序列预处理}
\subsubsection{TimeseriesGenerator}\label{preprocessing-sequence}

\begin{Shaded}
\begin{Highlighting}[]
\NormalTok{keras.preprocessing.sequence.TimeseriesGenerator(data, targets, length, 
sampling_rate}\OperatorTok{=}\DecValTok{1,}
\NormalTok{                                           stride}\OperatorTok{=}\DecValTok{1,}
\NormalTok{                                           start_index}\OperatorTok{=}\DecValTok{0,}
\NormalTok{                                           end_index}\OperatorTok{=}\VariableTok{None,}
\NormalTok{                                           shuffle}\OperatorTok{=}\VariableTok{False,}
\NormalTok{                                           reverse}\OperatorTok{=}\VariableTok{False,}
\NormalTok{                                           batch_size}\OperatorTok{=}\DecValTok{128}\NormalTok{)}
\end{Highlighting}
\end{Shaded}

用于生成批量时序数据的实用工具类。

这个类以一系列由相等间隔以及一些时间序列参数(例如步长、历史长度等)汇集的数据点作为输入,以生成用于训练/验证的批次数据。

\textbf{参数}

\begin{itemize}
\tightlist
\item
  data: 可索引的生成器(例如列表或 Numpy
  数组),包含连续数据点(时间步)。数据应该是 2D 的,且第 0
  个轴为时间维度。
\item
  targets: 对应于 \texttt{data} 的时间步的目标值。它应该与 \texttt{data}
  的长度相同。
\item
  length: 输出序列的长度(以时间步数表示)。
\item
  sampling\_rate: 序列内连续各个时间步之间的周期。对于周期 \texttt{r},
  时间步 \texttt{data{[}i{]}}, \texttt{data{[}i-r{]}}, \ldots{}
  \texttt{data{[}i\ -\ length{]}} 被用于生成样本序列。
\item
  stride: 连续输出序列之间的周期. 对于周期 \texttt{s}, 连续输出样本将为
  \texttt{data{[}i{]}}, \texttt{data{[}i+s{]}}, \texttt{data{[}i+2*s{]}}
  等。
\item
  start\_index: 在 \texttt{start\_index}
  之前的数据点在输出序列中将不被使用。这对保留部分数据以进行测试或验证很有用。
\item
  end\_index: 在 \texttt{end\_index}
  之后的数据点在输出序列中将不被使用。这对保留部分数据以进行测试或验证很有用。
\item
  shuffle: 是否打乱输出样本,还是按照时间顺序绘制它们。
\item
  reverse: 布尔值: 如果 \texttt{true},
  每个输出样本中的时间步将按照时间倒序排列。
\item
  batch\_size: 每个批次中的时间序列样本数(可能除最后一个外)。
\end{itemize}

\textbf{返回}

一个 \href{https://keras.io/zh/utils/\#sequence}{Sequence} 实例。

\textbf{例子}

\begin{Shaded}
\begin{Highlighting}[]
\ImportTok{from}\NormalTok{ keras.preprocessing.sequence }\ImportTok{import}\NormalTok{ TimeseriesGenerator}
\ImportTok{import}\NormalTok{ numpy }\ImportTok{as}\NormalTok{ np}

\NormalTok{data }\OperatorTok{=}\NormalTok{ np.array([[i] }\ControlFlowTok{for}\NormalTok{ i }\KeywordTok{in} \BuiltInTok{range}\NormalTok{(}\DecValTok{50}\NormalTok{)])}
\NormalTok{targets }\OperatorTok{=}\NormalTok{ np.array([[i] }\ControlFlowTok{for}\NormalTok{ i }\KeywordTok{in} \BuiltInTok{range}\NormalTok{(}\DecValTok{50}\NormalTok{)])}

\NormalTok{data_gen }\OperatorTok{=}\NormalTok{ TimeseriesGenerator(data, targets,}
\NormalTok{                               length}\OperatorTok{=}\DecValTok{10}\NormalTok{, sampling_rate}\OperatorTok{=}\DecValTok{2}\NormalTok{,}
\NormalTok{                               batch_size}\OperatorTok{=}\DecValTok{2}\NormalTok{)}
\ControlFlowTok{assert} \BuiltInTok{len}\NormalTok{(data_gen) }\OperatorTok{==} \DecValTok{20}

\NormalTok{batch_0 }\OperatorTok{=}\NormalTok{ data_gen[}\DecValTok{0}\NormalTok{]}
\NormalTok{x, y }\OperatorTok{=}\NormalTok{ batch_0}
\ControlFlowTok{assert}\NormalTok{ np.array_equal(x,}
\NormalTok{                      np.array([[[}\DecValTok{0}\NormalTok{], [}\DecValTok{2}\NormalTok{], [}\DecValTok{4}\NormalTok{], [}\DecValTok{6}\NormalTok{], [}\DecValTok{8}\NormalTok{]],}
\NormalTok{                                [[}\DecValTok{1}\NormalTok{], [}\DecValTok{3}\NormalTok{], [}\DecValTok{5}\NormalTok{], [}\DecValTok{7}\NormalTok{], [}\DecValTok{9}\NormalTok{]]]))}
\ControlFlowTok{assert}\NormalTok{ np.array_equal(y,}
\NormalTok{                      np.array([[}\DecValTok{10}\NormalTok{], [}\DecValTok{11}\NormalTok{]]))}
\end{Highlighting}
\end{Shaded}


\subsubsection{pad\_sequences}\label{pad_sequences}

\begin{Shaded}
\begin{Highlighting}[]
\NormalTok{keras.preprocessing.sequence.pad_sequences(sequences, maxlen}\OperatorTok{=}\VariableTok{None,}
\NormalTok{                                            dtype}\OperatorTok{=}\StringTok{'int32',}
\NormalTok{                                            padding}\OperatorTok{=}\StringTok{'pre',}
\NormalTok{                                            truncating}\OperatorTok{=}\StringTok{'pre',}
\NormalTok{                                            value}\OperatorTok{=}\FloatTok{0.0}\NormalTok{)}
\end{Highlighting}
\end{Shaded}

将多个序列截断或补齐为相同长度。

该函数将一个 \texttt{num\_samples} 的序列(整数列表)转化为一个 2D Numpy
矩阵,其尺寸为 \texttt{(num\_samples,\ num\_timesteps)}。
\texttt{num\_timesteps} 要么是给定的 \texttt{maxlen}
参数,要么是最长序列的长度。

比 \texttt{num\_timesteps} 短的序列将在末端以 \texttt{value} 值补齐。

比 \texttt{num\_timesteps}
长的序列将会被截断以满足所需要的长度。补齐或截断发生的位置分别由参数
\texttt{pading} 和 \texttt{truncating} 决定。

向前补齐为默认操作。

\textbf{参数}

\begin{itemize}
\tightlist
\item
  \textbf{sequences}: 列表的列表,每一个元素是一个序列。
\item
  \textbf{maxlen}: 整数,所有序列的最大长度。
\item
  \textbf{dtype}: 输出序列的类型。
\item
  \textbf{padding}: 字符串,`pre' 或 `post'
  ,表示长度不足时是在序列的前端补齐还是在后端补齐。
\item
  \textbf{truncating}: 字符串,`pre' 或 `post' ,移除长度大于
  \texttt{maxlen} 的序列的值,要么在序列前端截断,要么在后端。
\item
  \textbf{value}: 浮点数,表示用来补齐的值。
\end{itemize}

\textbf{返回}

\begin{itemize}
\tightlist
\item
  x: Numpy 矩阵,尺寸为 \texttt{(len(sequences),\ maxlen)}。
\end{itemize}

\textbf{异常}

\begin{itemize}
\tightlist
\item
  ValueError: 如果截断或补齐的值无效,或者序列条目的形状无效。
\end{itemize}


\subsubsection{skipgrams}\label{skipgrams}

\begin{Shaded}
\begin{Highlighting}[]
\NormalTok{keras.preprocessing.sequence.skipgrams(sequence, vocabulary_size, 
window_size}\OperatorTok{=}\DecValTok{4,}
\NormalTok{                                           negative_samples}\OperatorTok{=}\FloatTok{1.0,}
\NormalTok{                                           shuffle}\OperatorTok{=}\VariableTok{True,}
\NormalTok{                                           categorical}\OperatorTok{=}\VariableTok{False,}
\NormalTok{                                           sampling_table}\OperatorTok{=}\VariableTok{None,}
\NormalTok{                                           seed}\OperatorTok{=}\VariableTok{None}\NormalTok{)}
\end{Highlighting}
\end{Shaded}

生成 skipgram 词对。

该函数将一个单词索引序列(整数列表)转化为以下形式的单词元组:

\begin{itemize}
\tightlist
\item
  (单词, 同窗口的单词),标签为 1(正样本)。
\item
  (单词, 来自词汇表的随机单词),标签为 0(负样本)。
\end{itemize}

若要了解更多和 Skipgram 有关的知识,请参阅这份由 Mikolov 等人发表的经典论文:
\href{http://arxiv.org/pdf/1301.3781v3.pdf}{Efficient Estimation of Word Representations in Vector Space}

\textbf{参数}

\begin{itemize}
\tightlist
\item
  sequence: 一个编码为单词索引(整数)列表的词序列(句子)。如果使用一个
  \texttt{sampling\_table},词索引应该以一个相关数据集的词的排名匹配(例如,10
  将会编码为第 10 个最长出现的词)。注意词汇表中的索引 0
  是非单词,将被跳过。
\item
  vocabulary\_size: 整数,最大可能词索引 + 1
\item
  window\_size: 整数,采样窗口大小(技术上是半个窗口)。词 \texttt{w\_i}
  的窗口是 \texttt{{[}i\ -\ window\_size,\ i\ +\ window\_size+1{]}}。
\item
  negative\_samples: 大于等于 0 的浮点数。0 表示非负(即随机)采样。1
  表示与正样本数相同。
\item
  shuffle: 是否在返回之前将这些词语打乱。
\item
  categorical: 布尔值。如果 False,标签将为整数(例如
  \texttt{{[}0,\ 1,\ 1\ ..\ {]}}),如果 True,标签将为分类,例如
  \texttt{{[}{[}1,0{]},{[}0,1{]},{[}0,1{]}\ ..\ {]}}。
\item
  sampling\_table: 尺寸为 \texttt{vocabulary\_size} 的 1D 数组,其中第 i
  项编码了排名为 i 的词的采样概率。
\item
  seed: 随机种子。
\end{itemize}

\textbf{返回}

couples, labels: 其中 \texttt{couples} 是整数对,\texttt{labels} 是 0 或
1。

\textbf{注意}

按照惯例,词汇表中的索引 0 是非单词,将被跳过。


\subsubsection{make\_sampling\_table}\label{make_sampling_table}

\begin{Shaded}
\begin{Highlighting}[]
\NormalTok{keras.preprocessing.sequence.make_sampling_table(size, sampling_factor}\OperatorTok{=}\FloatTok{1e-05}\NormalTok{)}
\end{Highlighting}
\end{Shaded}

生成一个基于单词的概率采样表。

用来生成 \texttt{skipgrams} 的 \texttt{sampling\_table}
参数。\texttt{sampling\_table{[}i{]}} 是数据集中第 i
个最常见词的采样概率(出于平衡考虑,出现更频繁的词应该被更少地采样)。

采样概率根据 word2vec 中使用的采样分布生成:

\begin{Shaded}
\begin{Highlighting}[]
\NormalTok{p(word) }\OperatorTok{=}\NormalTok{ (}\BuiltInTok{min}\NormalTok{(}\DecValTok{1}\NormalTok{, sqrt(word_frequency }\OperatorTok{/}\NormalTok{ sampling_factor) }\OperatorTok{/}
\NormalTok{    (word_frequency }\OperatorTok{/}\NormalTok{ sampling_factor)))}
\end{Highlighting}
\end{Shaded}

我们假设单词频率遵循 Zipf 定律(s=1),来导出 frequency(rank)
的数值近似:

\texttt{frequency(rank)\ \textasciitilde{}\ 1/(rank\ *\ (log(rank)\ +\ gamma)\ +\ 1/2\ -\ 1/(12*rank))},其中
\texttt{gamma} 为 Euler-Mascheroni 常量。

\textbf{参数}

\begin{itemize}
\tightlist
\item
  \textbf{size}: 整数,可能采样的单词数量。
\item
  \textbf{sampling\_factor}: word2vec 公式中的采样因子。
\end{itemize}

\textbf{返回}

A 1D Numpy array of length size where the ith entry is the probability
that a word of rank i should be sampled.

一个长度为 \texttt{size} 大小的 1D Numpy 数组,其中第 i 项是排名为 i
的单词的采样概率。


\newpage
